\chapter{A: Some properties of $\Omega_{2m}(F,Q)$}
\label{AppA}
\stepcounter{appendixsection}

Let $V$ be a vector space over a field $F$ of dimension $n$. We assume that there is a 
non-singular quadratic form $Q$ defined on $V$. Denote by $\GO_n(F,Q)$ the group of 
non-singular linear transformations that preserve the
form $Q$. In case of characteristic $2$ we define the \textit{quasideterminant}
$\mathrm{qdeat}:\GO_n(F,Q) \rightarrow \mathbb{F}_2$ to be the map
\begin{equation}
    \operatorname{qdet}: g \mapsto \dim_F ( \im(\II - g) )\ \operatorname{mod}\ 2.
\end{equation}
Further, the group $\SO_n(F,Q)$ is the kernel of the (quasi)determinant map. Define the 
\textit{spinor norm} to be the homomorphism $\mathrm{sp} : \SO_n(F,Q) \rightarrow F^{\times} / 
(F^{\times})^2$. This homomorphism is defined in the following way. Any element of 
$\SO_n(F,Q)$ arising as a reflection in $v$ for some $v \in V$, is sent to the value
$Q(v)$ modulo $(F^{\times})^2$. This extends to a well-defined homomorphism. The subgroup
$\Omega_n(F, Q)$ of $\SO_n(F,Q)$ is defined as the kernel of spinor norm. If the characteristic 
of the field is not $2$, then there exists a double cover of $\Omega_n(F,Q)$, denoted as
$\Spin_n(F,Q)$. If $n = 2m$, $K:F$ is a 
quadratic Galois extension with $\s$ being the
nontrivial field automorphism fixing $F$
pointwise, and a maximal
totally isotropic subspace of $V$ has dimension
$m-1$, so that $V$ can be written as a direct
sum
\begin{equation}
V = \langle e_1, f_1 \rangle \oplus ...
\langle e_{m-1}, f_{m-1} \rangle \oplus 
W_2,	
\end{equation}
where $(e_1,f_1),...,(e_{m-1},f_{m-1})$ are
hyperbolic pairs and $W_2 \cong K$ with 
$Q_{W_2}(\lambda) = \lambda \lambda^{\s}$,
then we denote the group $\Omega_{2m}(F,Q)$
as $\Omega_{2m}^{-,K}(F)$. If $n = 2m$ and 
the dimension of a maximal totally isotropic 
subspace of $V$ is $m$, then we write
$\Omega_{2m}(F,Q)$ as $\Omega_{2m}^+(F)$.

This section is devoted to some of the private life of the group $\Omega_{2m}(F,Q)$.
Consider the vector space $V$ of dimension
$2m+2$ over $F$ with a non-singular quadratic form $Q$ defined on it. 
Let $f$ be a polar form of $Q$. 
Assuming that the Witt index of $Q$ is at least $1$, we can pick 
the basis $\mathcal{B} = \{v_1, w_1,...,w_{2m}, v_2\}$ in $V$ such that $(v_1,v_2)$ 
is a hyperbolic pair. Consider the decomposition $V = \langle v_1 \rangle \oplus
\langle w_1, ..., w_{2m} \rangle \oplus \langle v_2 \rangle$ and denote 
$W = \langle w_1, ..., w_{2m} \rangle$. Further, denote by $Q_W$ and $f_W$ the restrictions of 
$Q$ and $f$ on $W$.

\begin{lemma}
    \label{lemma:A_stabiliser_omega}
	The stabiliser in $\Omega_{2m+2}(F,Q)$ of the isotropic vector $v_1$ 
	is a subgroup of shape \mbox{$W\cn\Omega_{2m}(F,Q_W)$}, and the stabiliser of 
	the pair $(v_1,v_2)$ is a subgroup $\Omega_{2m}(F,Q_W)$.  
\end{lemma}

\begin{proof}
	Any element $g$ in $\Omega_{2m+2}(F,Q)$ which fixes $v_1$ also stabilises
    $\langle v_1 \rangle^{\perp}$, so it has the
    following form with respect to $\mathcal{B}$:
    \begin{equation*}
	[g]_{\mathcal{B}}=\left[
	    \begin{array}{c|c|c}
		1 & 0 & 0\  \\ \hline 
		 & & \\
		u_2^\T &\ \ \ \ \ A\ \ \ \ \  & 0\  \\ 
		 & & \\ \hline 
		\mu & u_1 & \lambda\ 
	    \end{array}
	\right],
    \end{equation*}
    where the matrix $A$ acts on the $2m$-dimensional subspace, 
    spanned by $\{w_1, ..., w_{2m}\}$. Such an element $g$ acts on $v_2$ as
    \begin{equation*}
	\begin{array}{r@{\;}c@{\;}l}
		v_2 & \mapsto & (\mu \mid u_1 \mid\ \lambda),
	\end{array}
    \end{equation*}
    and since the bilinear form $f$ is preserved we get
    \begin{equation*}
	1 = f(v_1,v_2) = f(v_1,\mu v_2) + 
	f(v_1, (0 \mid u_1 \mid\ 0) ) + \lambda f(v_1,v_2) = \lambda,
    \end{equation*}
    
    Since $(v_1,v_2)$ is a 
    hyperbolic pair, $f$ can be represented by the Gram matrix
    \begin{equation*}
	[f]_{\mathcal{B}} = \left[
	    \begin{array}{c|c|c}
		0 & 0 & 1  \\ \hline 
		 & & \\
		0 &\ \ \ \ \ B\ \ \ \ \  & 0 \\ 
		 & & \\ \hline 
		1 & 0 & 0 
	    \end{array}
	\right],
    \end{equation*}
    where $B$ is the matrix of $f_W$ with respect to the basis 
    $\{ w_1 , ..., w_{2m} \}$. 
    We explore the fact that an element in the stabiliser of $v_1$ preserves the 
    form $f$:
    \begin{multline*}
	\left[
	    \begin{array}{c|c|c}
		0 & 0 & 1  \\ \hline 
		 & & \\
		0 &\ \ \ \ \ B\ \ \ \ \  & 0 \\ 
		 & & \\ \hline 
		1 & 0 & 0 
	    \end{array}
	\right] =\\
	\end{multline*}
	\begin{multline*}
	= \left[
	    \begin{array}{c|c|c}
		1 & 0 & 0\  \\ \hline 
		 & & \\
		u_2^\T &\ \ \ \ \ A\ \ \ \ \  & 0\  \\ 
		 & & \\ \hline 
		\mu & u_1 & 1\ 
	    \end{array}
	\right]
	\left[
	    \begin{array}{c|c|c}
		0 & 0 & 1  \\ \hline 
		 & & \\
		0 &\ \ \ \ \ B\ \ \ \ \  & 0 \\ 
		 & & \\ \hline 
		1 & 0 & 0 
	    \end{array}
	\right]
	\left[
	    \begin{array}{c|c|c}
		1 & u_2 & \mu\  \\ \hline 
		 & & \\
		0 &\ \ \ \ \ A^\T\ \ \ \ \  & u_1^\T  \\ 
		 & & \\ \hline 
		0 & 0 & 1\  
	    \end{array}
	\right] =
    \end{multline*}
    \begin{equation*}
	= \left[
	    \begin{array}{c|c|c}
		0 & 0 & 1\ \\ \hline
		& & \\
		0 &\ \ \ \ ABA^\T\ \ \ \ & ABu_1^\T + u_2^\T \\
		& & \\ \hline
		1 & u_2+u_1 BA^\T & 2\mu + u_1 B u_1^\T
	    \end{array}
	\right],
    \end{equation*}
    so we notice that $ABA^\T = B$. Furthermore, since
    $(0\mid v \mid 0) [g]_{\mathcal{B}} = (0 \mid vA \mid 0)$, where $v \in W$, we obtain
    \begin{equation*}
	Q_W(vA) = Q([0 \mid vA \mid 0]) = Q([0 \mid v \mid 0]) =
	    Q_W(v),
    \end{equation*}
    so $A$ is an element of $\GO_{2m}(F,Q)$.
    Additionally, $u_2 = -u_1BA^\T$ and we see that $u_2$ is uniquely determined by $u_1$. 
    From the bottom right corner of the resulting matrix we obtain
    $\mu = -Q(u_1)$ in odd characteristic. In case of characteristic $2$ we can 
    explore the quadratic form again:
    \begin{multline*}
	0 = Q(v_2) = Q(v_2 [g]_{\mathcal{B}}) = Q( (\mu \mid u_1 \mid 1) ) = \\
	= Q( (\mu \mid u_1 \mid 0) ) + Q(v_2) + f( (\mu\mid u_1 \mid 0), v_2)
	= Q(u_1) + \mu.
    \end{multline*}
    
    Consider the decomposition
    \begin{equation*}
	\left[
	    \begin{array}{c|c|c}
		1 & 0 & 0\  \\ \hline 
		 & & \\
		-ABu_1^\T &\ \ \ \ \ A\ \ \ \ \  & 0\  \\ 
		 & & \\ \hline 
		-Q(u_1) & u_1 & 1
	    \end{array}
	\right] = \left[
	    \begin{array}{c|c|c}
		1 & 0 & 0 \\ \hline 
		 & & \\
		0 &\ \ \ \ \ A\ \ \ \ \  & 0  \\ 
		 & & \\ \hline 
		0 & 0 & 1 
	    \end{array}
	\right]
	\left[
	    \begin{array}{c|c|c}
		1 & 0 & 0\  \\ \hline 
		 & & \\
		-Bu_1^\T &\ \ \ \ \ \II_{2m}\ \ \ \ \  & 0\  \\ 
		 & & \\ \hline 
		-Q(u_1) & u_1 & 1
	    \end{array}
	\right].
    \end{equation*}
    The matrices of the form
    \begin{equation*}
	C_{u_1} = \left[
	    \begin{array}{c|c|c}
		1 & 0 & 0\  \\ \hline 
		 & & \\
		-Bu_1^\T &\ \ \ \ \ \II_{2m}\ \ \ \ \  & 0\  \\ 
		 & & \\ \hline 
		-Q(u_1) & u_1 & 1
	    \end{array}
	\right]
    \end{equation*}
    generate an elementary abelian group isomorphic to $W$ (as abelian groups). 
    Indeed, since the product
    of two such matrices is given by
    \begin{multline*}
	\left[
	    \begin{array}{c|c|c}
		1 & 0 & 0\  \\ \hline 
		 & & \\
		-Bu^\T &\ \ \ \ \ \II_{2m}\ \ \ \ \  & 0\  \\ 
		 & & \\ \hline 
		-Q(u) & u & 1
	    \end{array}
	\right]\left[
	    \begin{array}{c|c|c}
		1 & 0 & 0\  \\ \hline 
		 & & \\
		-Bv^\T &\ \ \ \ \ \II_{2m}\ \ \ \ \  & 0\  \\ 
		 & & \\ \hline 
		-Q(v) & v & 1 
	    \end{array}
	\right]= \\
	=\left[
	    \begin{array}{c|c|c}
		1 & 0 & 0\  \\ \hline 
		 & & \\
		-B(u+v)^\T &\ \ \ \ \ \II_{2m}\ \ \ \ \  & 0\  \\ 
		 & & \\ \hline 
		-Q(u+v) & u+v & 1
	    \end{array}
	\right],
    \end{multline*}
    we see that the set of these matrices is closed under multiplication 
    and moreover any two such matrices commute.
    
    To show that the matrix $A$ is an element of $\Omega_{2m}(F,Q)$,
    we use Proposition 1.6.11 from \cite{BrayHoltRD} to calculate the spinor 
    norm and, in case of characteristic $2$, the quasideterminant of the
    matrices $C_{u_1}$. Note that $\det(C_{u_1}) = \det([g]_{\mathcal{B}}) = 1$.
    Consider the matrix
    \begin{equation*}
	\II-C_{u_1} = \left[
	    \begin{array}{c|c|c}
		0 & 0 & 0  \\ \hline 
		 & & \\
		Bu_1^\T &\ \ \ \ \ 0\ \ \ \ \  & 0\  \\ 
		 & & \\ \hline 
		Q(u_1) & -u_1 & 0
	    \end{array}
	\right].
    \end{equation*}
    For a vector $v$ we denote by $[v]_i$ its $i$-th component. Now, 
    if $u_1 = 0$, then $\II-C_{u_1}$ has rank $0$, whereas if $u_1 \neq 0$, then
    there is an index $i$ such that $[Bu_1^\T]_i \neq 0$ and it follows that
    the rank of $\II-C_{u_1}$ in this case is $2$. Consequently, $k=\rank(\II-C_{u_1})$ is
    even, and so by the Proposition 1.6.11 in \cite{BrayHoltRD} the quasideterminant of $C_{u_1}$
    is $1$. Further, if $D$ is a $k\times (2m+2)$ matrix whose rows are the basis
    elements of a complement of the nullspace of $\II-C_{u_1}$, then the spinor norm of
     $C_{u_1}$ is $1$
    if $\det(D(\II-C_{u_1}) [f]_{\mathcal{B}} D^\T)$ is a square in $F$. If $u_1 \neq 0$,
    then the complement of the nullspace of $I-C_{u_1}$ has the basis
    $\{ w_i, v_1 \}$, where the index $i$ is such that $[Bu_1^\T]_i \neq 0$. 
    The matrix $D$ has the following form:
    \begin{equation*}
	D = \begin{bmatrix}
	    0 & 0 & \cdots & 0 & 1 & 0 & \cdots & 0 & 0 \\
	    0 & 0 & \cdots & 0 & 0 & 0 & \cdots & 0 & 1
	\end{bmatrix},
    \end{equation*}
    where $1$ in the first row is in the $(i+1)$-st position. We calculate
    \begin{equation*}
	D(\II-C_{u_1}) = \left[
	    \begin{array}{c|c|c}
		\alpha & 0 & 0 \\ \hline
		Q(u_1) & -u_1 & 0
	    \end{array}
	\right],\ 
	[f]_{\mathcal{B}} D^\T = \begin{bmatrix}
	    0 & 1\  \\
	    B_{1,i} & 0\  \\
	    \vdots & \vdots \\
	    B_{2m,i} & 0\  \\
	    0 & 0
	\end{bmatrix},
    \end{equation*}
    where $\alpha = [Bu_1^\T]_i = [u_1B]_i$. Finally, 
    \begin{equation*}
	D(\II-C_{u_1}) [f]_{\mathcal{B}} D^\T = \begin{bmatrix}
	    0 & \alpha \\
	    -\alpha & Q(u_1)
	\end{bmatrix},
    \end{equation*}
    so $\det(D(\II-C_{u_1})FD^\T) = \alpha^2$ as needed. Since the quasideterminant and 
    the spinor norm are multiplicative (Theorems 11.43 and 11.50 in \cite{Taylor}),
    and $g \in \Omega_{2m+2}(F,\hat{Q})$, we conclude that $A$ acts on $W$ as
    an element of $\Omega_{2m}(F,Q)$ and it follows that the stabiliser
    of $v_1$ in $\Omega_{2m}(F,\hat{Q})$ is indeed a subgroup of shape
    $W\cn\Omega_{2m}(F,Q)$.
    
    Lastly, if we stabilise $v_1$ and $v_2$ simultaneously, a general element in the
    stabiliser takes the form
    \begin{equation*}
	\left[
	    \begin{array}{c|c|c}
		1 & 0 & 0 \\ \hline 
		 & & \\
		0 &\ \ \ \ \ A\ \ \ \ \  & 0 \\ 
		 & & \\ \hline 
		0 & 0 & 1
	    \end{array}
	\right],
    \end{equation*}
    so the stabiliser of the pair $(v_1,v_2)$ is $\Omega_{2m}(F,Q)$. 
\end{proof}

Witt's lemma tells us that the group $\GO_{2m}(F,Q)$ acts transitively 
on the non-zero vectors of each norm. The following result allows us to use the fact
that the same is true for $\Omega_{2m}(F,Q)$ in case when Q is of Witt index at least $1$.

\begin{lemma}
    \label{lemma:A_omega_transitive}
    The group $\Omega_{2m+2}(F,Q)$, where $Q$ is of Witt index at least $2$, 
    acts transitively on
    \begin{equation*}
	O_{\lambda} = \left\{ \  v \in V\ \big|\ Q(v) = \lambda,\ v \neq 0 \ \right\}
    \end{equation*}
    for each value of $\lambda \in F$.
\end{lemma}

\begin{proof}
    Suppose $u, v \in O_{\lambda}$. Since by Witt's lemma $\GO_{2m+2}(F,Q)$
    acts transitively on $O_{\lambda}$, there is an element 
    $g \in \GO_{2m+2}(F,Q)$ which sends $u$ to $v$. Consider the basis 
    for $V$, $\mathcal{B} = \{v_1, w_1, ..., w_{2m-2}, v_1\}$ such that as before
    $(v_1,v_2)$ is a hyperbolic pair and $u \in \langle v_1, v_2 \rangle$, and so
    $V = \langle v_1 \rangle \oplus \langle w_1, ..., w_{2m-2} \rangle
    \oplus \langle v_2 \rangle$. Suppose $h$ is an element in the stabiliser of $(v_1,v_2)$.
    As a consequence, $h$ stabilises $u$ and with respect to $\mathcal{B}$ it has 
    the form
    \begin{equation*}
	[h]_{\mathcal{B}} = \left[
	    \begin{array}{c|c|c}
		1 & 0 & 0 \\ \hline 
		 & & \\
		0 &\ \ \ \ \ A\ \ \ \ \  & 0 \\ 
		 & & \\ \hline 
		0 & 0 & 1
	    \end{array}
	\right],
    \end{equation*}
    where the matrix $A$ acts on \mbox{$W=\langle w_1, ..., w_{2m-2} \rangle$} as an
     element of $\GO_{2m}(W,Q_W)$. If the determinant of $g$ is $1$,
    then we may take
    \begin{equation*}
	A = \begin{bmatrix}
	    \mu & 0 & 0 & \cdots & 0 \\
	    0 & \mu^{-1} & 0 & \cdots & 0 \\
	    0 & 0 & 1 & \cdots & 0 \\
	    \vdots & \vdots & \vdots & \ddots & \vdots \\
	    0 & 0 & 0 & \cdots & 1
	\end{bmatrix},
    \end{equation*}
    where $\mu \in F$. On the other hand, if $\det(g) = -1$, then we take		
    \begin{equation*}
	A = \begin{bmatrix}
	    0 & \mu^{-1} & 0 & \cdots & 0 \\
	    \mu & 0 & 0 & \cdots & 0 \\
	    0 & 0 & 1 & \cdots & 0 \\
	    \vdots & \vdots & \vdots & \ddots & \vdots \\
	    0 & 0 & 0 & \cdots & 1
	\end{bmatrix}
    \end{equation*}	
    instead, so we can always get $\det(hg) = 1$. Note that the latter choice of 
    $A$ also adjusts the quasideterminant in characteristic $2$ as the 
    rank of $\II-[h]_{\mathcal{B}}$ in this case is odd. 
    Finally, choosing $\mu$ accordingly 
    we can ensure that the spinor norm of $hg$ is $1$, i.e. $hg \in \Omega_{2m+2}(F,Q)$. 
\end{proof}

The following theorems are explicitly used in the constructions of certain orthogonal subgroups
of $\EE_6(F)$ and ${}^2\EE_6^K(F)$. 

\begin{theorem}
	\label{theorem:A_omega_maximal}
	Let $Q_W$ be of Witt index at least $1$. The subgroup $\Omega_{2m}(F,Q_W)$ is maximal
	in $W\cn\Omega_{2m}(F,Q_W)$.
\end{theorem}

\begin{proof}
	Recall that $v_2 \in V$ is mapped under the action of 
    $\mathcal{G} = W\cn\Omega_{2m}(F,Q_W)$ to a vector of the
    form $(-Q_W(u) \mid u \mid 1)7$, where $u$ is an element of $W$. 
    Since the stabiliser of $v_2$ in $\mathcal{G}$ is $\Omega_{2m}(F,Q_W)$, 
    we conclude that 
    the orbit of $v_2$ under the action of $\mathcal{G}$ is the following set:
    \begin{equation*}
	\Orb_{\mathcal{G}}(v_2) = \left\{\  \left( -Q_W(u) \mid  u \mid 1
	 \right)\ \big|\ u \in W \ \right\}.
    \end{equation*}
    Since the elements of this orbit are in one-to-one correspondence with the
    cosets of $\Omega_{2m}(F,Q_W)$ in $\mathcal{G}$, 
    it is enough to show the primitive
    action on $\Orb_{\mathcal{G}}(v_2)$. 
    
    Consider the action of $\mathcal{G}$ on 
    $\Orb_{\mathcal{G}}(v_2)$. A general element in $\mathcal{G}$ 
    acts on the elements of $\Orb_{\mathcal{G}}(v_2)$ in the following way:
    \begin{multline*}
	( -Q_W(u) \mid  u \mid 1 ) \ \mapsto \ 
	( -Q_W(u) - uABv^\T - Q(v) \mid
	    uA + v \mid 1) = \\
	    = (-Q_W(uA + v) \mid uA + v \mid 1).
    \end{multline*}
    Note that $uABv^\T = f_W( uA, v )$. 
    We see that this action is isomorphic to the action on $W$ defined by
	$u \mapsto uA + v$,
    where $u,v \in W$. In case when $A$ is the identity matrix, this map
    is a translation. On the other hand, taking $v = 0$, we obtain the
    action of $\Omega_{2m}(F,Q_W)$. 
    Denote the group generated by the described action on $W$ as
    $\mathrm{A\Omega}_{2m}(F,Q_W)$.
    
    Since $Q_W$ is of Witt index at least $1$, we may choose a hyperbolic pair 
    $(u_1,u_2)$ in $W$ such that $W = \langle u_1, u_2 \rangle \oplus U$, where 
    $U = \langle u_1, u_2 \rangle^{\perp}$. We aim to show that any 
    $\mathrm{A\Omega}_{2m}(F,Q_W)$-congruence on $W$ is trivial (and hence, the 
    action is primitive). Suppose $w_1 \sim w_2$, where $w_1,w_2 \in W$ and 
    $\sim$ is some congruence relation preserved by $\mathrm{A\Omega}_{2m}(F,Q_W)$.
    It follows that $w_1 - w_2 \sim 0$, and so we may start with $v \sim 0$ for some
    $v \in W$. We distinguish two cases.
    
    First, if $Q_W(v) = 0$, then since $\Omega_{2m}(F,Q_W)$ acts transitively on isotropic
    vectors in $W$, we get $u_1 \sim 0$, $u_2 \sim 0$, and also $-\lambda u_2 \sim 0$ for any 
    $\lambda \in F$. Since $\sim$ is a congruence relation, it is transitive and so
    $u_1 \sim -\lambda u_2$, from which it follows $u_1 + \lambda u_2 \sim 0$. Now, 
    $Q_W(u_1 + \lambda u_2) = \lambda$, and $\lambda$ is an arbitrary field element,
    so $\sim$ is trivial.
    
    Next, if $Q_W(v) = \lambda$ for some non-zero $\lambda \in F$, we consider two vectors
    $w_1 = u_1 + \lambda u_2$ and $w_2 = u_1 + (\lambda - Q_W(u))u_2 + u$ for some $u \in U$. 
    Note that $Q_W(w_1) = Q_W(w_2) = \lambda$, so since $\Omega_{2m}(F,Q_W)$
    acts transitively on the vectors of norm $\lambda$, we obtain
    $w_1 \sim 0$ and $w_2 \sim 0$, from which it immediately follows that 
    $w_1 \sim w_2$, and further $w_1 - w_2 \sim 0$. 
    We find $Q_W(w_1 - w_2) = Q_W(u - Q(u)u_2) = Q_W(u)$, so in fact we have
    $u \sim 0$ for some $u \in U$. From $Q(u_1+u) = Q(u)$ it follows that 
    $u_1 + u \sim 0$, and so by transitivity $u_1+u \sim u$. We subtract $u$ from both
    sides to obtain $u_1 \sim 0$, which is covered by the previous case. 
\end{proof}

As we already know from Lemma \ref{lemma:A_stabiliser_omega}, the stabiliser in $\Omega_{2m+2}(F,Q)$
of an isotropic vector $v_1 \in V$ is a subgroup of shape $W\cn\Omega_{2m}(F,Q_W)$. We also find that
every proper 
subgroup of $\Omega_{2m+2}(F,Q)$, containing $W\cn\Omega_{2m}(F,Q_W)$ as a subgroup, stabilises 
the $1$-space spanned by $v_1$.

\begin{theorem}
	\label{theorem:A_space_stab}
	Let $Q_W$ be of Witt index at least $2$. Any subgroup $H$ such that
	\begin{equation}
		W\cn\Omega_{2m}(F,Q_W) \leqslant H < \Omega_{2m+2}(F,Q),
	\end{equation}
	stabilises the $1$-space $\langle v_1 \rangle$. 
\end{theorem}

\begin{proof}
	Let $G = \Omega_{2m+2}(F,Q)$. We aim to prove that if $v_1 \sim v$ for some $v \in V$, where
	$\sim$ is any non-trivial $G$-congruence, then $v = \lambda v_1$ for some $\lambda \in F$. 
	For the sake of finding a contradiction, suppose that $v = \lambda v_1 + u + \mu v_2$,
	where $u + \mu v_2 \neq 0$, i.e. either $u \neq 0$ or $\mu \neq 0$. We distinguish two cases.
	
	First, if $\mu = 0$, then $v_1 \sim v$, where $v = \lambda v_1 + u$ with $0\neq u \in W$.
	We have $0 = Q(v) = Q_W(u)$. 
	Recall that $W\cn \Omega_{2m}(F,Q_W)$ is the stabiliser in $\Omega_{2m+2}(F, Q)$ of $v_1$,
	so with respect to the familiar basis $\mathcal{B} = \{v_1, w_1 ,..., w_{2m}, v_2\}$,
	its general element has the form 
	\begin{equation*}
	[g]_{\mathcal{B}}=\left[
	    \begin{array}{c|c|c}
		1 & 0 & 0\  \\ \hline 
		 & & \\
		u_2^\T &\ \ \ \ \ A\ \ \ \ \  & 0\  \\ 
		 & & \\ \hline 
		\nu & u_1 & 1\ 
	    \end{array}
	\right],
	\end{equation*}
	where $u_2 = -u_1 BA^{\T}$. Now, $g$ maps $v$ to a vector of the form \mbox{$v^g = 
	(\lambda + u u_2^{\T}) v_1 + uA$}. Since $u u_2^{\T} = 
	-u A B u_1^{\T} = f_W(u, u_1)$, we get \mbox{$v^g = (\lambda - f_W(u, u_1))v_1 + uA$}.
	Setting $A = \II_{2m}$, we see that it is possible to send $v$ to a vector
	of the form $\alpha v_1 + u$ for any $\alpha \in F$. On the other hand, taking $u_1 = 0$,
	we obtain the action on \mbox{$W = \langle w_1, ..., w_{2m} \rangle$} 
	of $\Omega_{2m}(F,Q_W)$, which is 
	transitive on the isotropic vectors in $W$. We have shown that $v_1 \sim v$ for any $v$ of
	the form $\alpha v_1 + u$ for arbitrary $\alpha \in F$ and $0\neq u \in W$. 
	The group $\Omega_{2m+2}(F,Q)$	in 
	its turn is transitive on the isotropic vectors in $V$, so there exists an element
	\mbox{$h \in \Omega_{2m+2}(F,Q)$} such that $v_1^h = v_2$ and $v^h \in W \oplus \langle v_2 \rangle$. 
	It follows that $v_2 \sim v^h$ and so, by transitivity of $\sim$ we find that 
	$v_1 \sim v_2$. Finally, it is easy to derive the congruences of the form 
	$v_1 \sim \beta v_1$ and $v_2 \sim \beta v_2$ for any non-zero $\beta \in F$. Indeed, 
	there exists an element \mbox{$g \in  \Omega_{2m+2}(F,Q)$} such that $v_1^g = \beta v_1$ for some non-zero
	$\beta \in F$. We have 
\mbox{$\beta v_2 \sim \alpha \beta v_1 + u^g$}, so \mbox{$v_1 \sim \beta v_1$}. Similarly we obtain
	$v_2 \sim \beta v_2$ for any non-zero $\beta \in F$, and so $\sim$ is universal, contradiction.

The case $\mu = 0$ can be established with 
essentially similar reasoning. 
	
	
\end{proof}

% ------------------------------------------------------------------------

%%% Local Variables: 
%%% mode: latex
%%% TeX-master: "../thesis"
%%% End: 
