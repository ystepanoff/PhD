
% ------------------------------------------------------------------------
\chapter{A: Some properties of $\Omega_{2m}(F,Q)$}
\label{AppA}
\stepcounter{appendixsection}

Let $V$ be a vector space over a field $F$ of dimension $n$. We assume that there is a 
non-singular quadratic form $Q$ defined on $V$. Denote by $\GO_n(F,Q)$ the group of 
non-singular linear transformations that preserve the
form $Q$. In case of characteristic $2$ we define the \textit{quasideterminant}
$\mathrm{qdet}:\GO_n(F,Q) \rightarrow \mathbb{F}_2$ to be the map
\begin{equation}
    \operatorname{qdet}: g \mapsto \dim_F ( \im(\II - g) )\ \operatorname{mod}\ 2.
\end{equation}
Further, the group $\SO_n(F,Q)$ is the kernel of the (quasi)determinant map. Define the 
\textit{spinor norm} to be the homomorphism $\mathrm{sp} : \SO_n(F,Q) \rightarrow F^{\times} / 
(F^{\times})^2$. This homomorphism is defined in the following way. Any element of 
$\SO_n(F,Q)$ arising as a reflection in $v$ for some $v \in V$, is sent to the value
$Q(v)$ modulo $(F^{\times})^2$. This extends to a well-defined homomorphism. The subgroup
$\Omega_n(F, Q)$ of $\SO_n(F,Q)$ is defined as the kernel of spinor norm. If the characteristic 
of the field is not $2$, then there exists a double cover of $\Omega_n(F,Q)$, denoted as
$\Spin_n(F,Q)$. 

This section is devoted to some of the private life of the group $\Omega_{2m}(F,Q)$.
Consider the vector space $V$ of dimension
$2m+2$ over $F$ with a non-singular quadratic form $Q$ defined on it. 
Let $f$ be a polar form of $Q$. 
Assuming that the Witt index of $Q$ is at least $1$, we can pick 
the basis $\mathcal{B} = \{v_1, w_1,...,w_{2m}, v_2\}$ in $V$ such that $(v_1,v_2)$ 
is a hyperbolic pair. Consider the decomposition $V = \langle v_1 \rangle \oplus
\langle w_1, ..., w_{2m} \rangle \oplus \langle v_2 \rangle$ and denote 
$W = \langle w_1, ..., w_{2m} \rangle$. Further, denote by $Q_W$ the restriction of 
$Q$ on $W$.

\begin{lemma}
    \label{lemma:1_stabiliser_omega}
	The stabiliser in $\Omega_{2m+2}(F,Q)$ of the vector $v_1$ 
	is a subgroup of shape $W\cn\Omega_{2m}(F,Q_W)$, and the stabiliser of 
	the pair $(v_1,v_2)$ is a subgroup $\Omega_{2m}(F,Q_W)$.  
\end{lemma}

\begin{proof}
	Any element in $\Omega_{2m+2}(F,Q)$ which fixes $v_1$ also stabilises
    $\langle v_1 \rangle^{\perp}$, so it has the
    following form:
    \begin{equation*}
	\hat{A}=\left(
	    \begin{array}{c|c|c}
		1 & 0 & 0\  \\ \hline 
		 & & \\
		u_2^\T &\ \ \ \ \ A\ \ \ \ \  & 0\  \\ 
		 & & \\ \hline 
		\mu & u_1 & \lambda\ 
	    \end{array}
	\right),
    \end{equation*}
    where the matrix $A$ acts on the $2m$-dimensional subspace, 
    spanned by $\{w_1, ..., w_{2m}\}$. 
    Suppose $u_1 = (\nu_1, ...,  \nu_{2m})$, so such an element acts on $v_1$ as
    \begin{equation*}
	\begin{array}{rcl}
	    v_1 & \mapsto & \mu v_0 + \sum\limits_{i=1}^{2m} \nu_i w_i + \lambda v_1,
	\end{array}
    \end{equation*}
    but since the bilinear form $f$ is preserved we get
    \begin{equation*}
	1 = f(v_0,v_1) = f(v_0,\mu v_0) + \sum\limits_{i=1}^{2m} f(v_0,\nu_i w_i) + \lambda
	f(v_0,v_1) = \lambda,
    \end{equation*}
    i.e. $\lambda = 1$. Consider the decomposition $F^{2m+2} = \langle v_0 \rangle \oplus
    \langle w_1, ..., w_{2m} \rangle \oplus \langle v_1 \rangle$ and denote by $Q$ and
    $\beta$ the quadratic and bilinear forms
    on the subspace $F^{2m} = \langle w_1,...,w_{2m} \rangle$ obtained as the restriction of 
    $\hat{Q}$ and $f$ respectively.
    
    Since $(v_0,v_1)$ is a 
    hyperbolic pair, the form $f$ on $F^{2m+2}$ can be represented by the Gram matrix
    \begin{equation*}
	[f]_{\mathcal{B}} = \left(
	    \begin{array}{c|c|c}
		0 & 0 & 1  \\ \hline 
		 & & \\
		0 &\ \ \ \ \ B\ \ \ \ \  & 0 \\ 
		 & & \\ \hline 
		1 & 0 & 0 
	    \end{array}
	\right),
    \end{equation*}
    where $B$ is the matrix of $\beta$ with respect to the basis 
    $\{ w_1 , ..., w_{2m} \}$. 
    We explore the fact that an element in the stabiliser of $v_0$ preserves the 
    form $f$:
    \begin{multline*}
	\left(
	    \begin{array}{c|c|c}
		0 & 0 & 1  \\ \hline 
		 & & \\
		0 &\ \ \ \ \ B\ \ \ \ \  & 0 \\ 
		 & & \\ \hline 
		1 & 0 & 0 
	    \end{array}
	\right) =\\
	= \left(
	    \begin{array}{c|c|c}
		1 & 0 & 0\  \\ \hline 
		 & & \\
		u_2^\T &\ \ \ \ \ A\ \ \ \ \  & 0\  \\ 
		 & & \\ \hline 
		\mu & u_1 & 1\ 
	    \end{array}
	\right)
	\left(
	    \begin{array}{c|c|c}
		0 & 0 & 1  \\ \hline 
		 & & \\
		0 &\ \ \ \ \ B\ \ \ \ \  & 0 \\ 
		 & & \\ \hline 
		1 & 0 & 0 
	    \end{array}
	\right)
	\left(
	    \begin{array}{c|c|c}
		1 & u_2 & \mu\  \\ \hline 
		 & & \\
		0 &\ \ \ \ \ A^\T\ \ \ \ \  & u_1^\T  \\ 
		 & & \\ \hline 
		0 & 0 & 1\  
	    \end{array}
	\right) =
    \end{multline*}
    \begin{equation*}
	= \left(
	    \begin{array}{c|c|c}
		0 & 0 & 1\ \\ \hline
		& & \\
		0 &\ \ \ \ ABA^\T\ \ \ \ & ABu_1^\T + u_2^\T \\
		& & \\ \hline
		1 & u_2+u_1 BA^\T & 2\mu + u_1 B u_1^\T
	    \end{array}
	\right),
    \end{equation*}
    so we notice that $ABA^\T = B$. Furthermore, since
    $(0\mid v \mid 0) \hat{A} = (0 \mid vA \mid 0)$, where $v \in F^{2m}$, we obtain
    \begin{equation*}
	Q(vA) = \hat{Q}((0 \mid vA \mid 0) ) = \hat{Q}((0 \mid v \mid 0) ) =
	    Q(v),
    \end{equation*}
    so $A$ is an element of $\GO_{2m}(F,Q)$.
    Additionally, $u_2 = -u_1BA^\T$ and we see that $u_2$ is uniquely determined by $u_1$. 
    From the bottom right corner of the resulting matrix we obtain
    $\mu = -Q(u_1)$ in odd characteristic. In case of characteristic $2$ we can 
    explore the quadratic form again:
    \begin{multline*}
	0 = \hat{Q}(v_1) = \hat{Q}(v_1 \hat{A}) = \hat{Q}( (\mu \mid u_1 \mid 1) ) = \\
	= \hat{Q}( (\mu \mid u_1 \mid 0) ) + \hat{Q}(v_1) + f( (\mu\mid u_1 \mid 0), v_1)
	= Q(u_1) + \mu.
    \end{multline*}
    
    Consider the decomposition
    \begin{equation*}
	\left(
	    \begin{array}{c|c|c}
		1 & 0 & 0\  \\ \hline 
		 & & \\
		-ABu_1^\T &\ \ \ \ \ A\ \ \ \ \  & 0\  \\ 
		 & & \\ \hline 
		-Q(u_1) & u_1 & 1
	    \end{array}
	\right) = \left(
	    \begin{array}{c|c|c}
		1 & 0 & 0 \\ \hline 
		 & & \\
		0 &\ \ \ \ \ A\ \ \ \ \  & 0  \\ 
		 & & \\ \hline 
		0 & 0 & 1 
	    \end{array}
	\right)
	\left(
	    \begin{array}{c|c|c}
		1 & 0 & 0\  \\ \hline 
		 & & \\
		-Bu_1^\T &\ \ \ \ \ \II_{2m}\ \ \ \ \  & 0\  \\ 
		 & & \\ \hline 
		-Q(u_1) & u_1 & 1
	    \end{array}
	\right),
    \end{equation*}
    The matrices of the form
    \begin{equation*}
	C_{u_1} = \left(
	    \begin{array}{c|c|c}
		1 & 0 & 0\  \\ \hline 
		 & & \\
		-Bu_1^\T &\ \ \ \ \ \II_{2m}\ \ \ \ \  & 0\  \\ 
		 & & \\ \hline 
		-Q(u_1) & u_1 & 1
	    \end{array}
	\right)
    \end{equation*}
    generate an elementary abelian group $F^{2m}$. Indeed, since the product
    of two such matrices is given by
    \begin{multline*}
	\left(
	    \begin{array}{c|c|c}
		1 & 0 & 0\  \\ \hline 
		 & & \\
		-Bu^\T &\ \ \ \ \ \II_{2m}\ \ \ \ \  & 0\  \\ 
		 & & \\ \hline 
		-Q(u) & u & 1
	    \end{array}
	\right)\left(
	    \begin{array}{c|c|c}
		1 & 0 & 0\  \\ \hline 
		 & & \\
		-Bv^\T &\ \ \ \ \ \II_{2m}\ \ \ \ \  & 0\  \\ 
		 & & \\ \hline 
		-Q(v) & v & 1 
	    \end{array}
	\right)= \\
	=\left(
	    \begin{array}{c|c|c}
		1 & 0 & 0\  \\ \hline 
		 & & \\
		-B(u+v)^\T &\ \ \ \ \ \II_{2m}\ \ \ \ \  & 0\  \\ 
		 & & \\ \hline 
		-Q(u+v) & u+v & 1
	    \end{array}
	\right),
    \end{multline*}
    we see that the set of these matrices is closed under multiplication 
    and moreover any two such matrices commute.
    
    To show that the matrix $A$ is an element of $\Omega_{2m}(F,Q)$,
    we use Proposition 1.6.11 from \cite{BrayHoltRD} to calculate the spinor 
    norm and, in case of characteristic $2$, the quasideterminant of the
    matrices $C_{u_1}$. Note that $\det(C_{u_1}) = \det(\hat{A}) = 1$.
    Consider the matrix
    \begin{equation*}
	\II-C_{u_1} = \left(
	    \begin{array}{c|c|c}
		0 & 0 & 0  \\ \hline 
		 & & \\
		Bu_1^\T &\ \ \ \ \ 0\ \ \ \ \  & 0\  \\ 
		 & & \\ \hline 
		Q(u_1) & -u_1 & 0
	    \end{array}
	\right).
    \end{equation*}
    For a vector $v$ we denote by $[v]_i$ its $i$-th component. Now, 
    if $u_1 = 0$, then $\II-C_{u_1}$ has rank $0$, whereas if $u_1 \neq 0$, then
    there is an index $i$ such that $[Bu_1^\T]_i \neq 0$ and it follows that
    the rank of $\II-C_{u_1}$ in this case is $2$. Consequently, $k=\rank(\II-C_{u_1})$ is
    even, and so by the Proposition 1.6.11 in \cite{BrayHoltRD} the quasideterminant of $C_{u_1}$
    is $1$. Further, if $D$ is a $k\times (2m+2)$ matrix whose rows are the basis
    elements of a complement of the nullspace of $\II-C_{u_1}$, then the spinor norm of
     $C_{u_1}$ is $1$
    if $\det(D(\II-C_{u_1}) [f]_{\mathcal{B}} D^\T)$ is a square in $F$. If $u_1 \neq 0$,
    then the complement of the nullspace of $I-C_{u_1}$ has the basis
    $\{ w_i, v_1 \}$, where the index $i$ is such that $[Bu_1^\T]_i \neq 0$. 
    The matrix $D$ has the following form:
    \begin{equation*}
	D = \begin{pmatrix}
	    0 & 0 & \cdots & 0 & 1 & 0 & \cdots & 0 & 0 \\
	    0 & 0 & \cdots & 0 & 0 & 0 & \cdots & 0 & 1
	\end{pmatrix},
    \end{equation*}
    where $1$ in the first row is in the $(i+1)$-st position. We calculate
    \begin{equation*}
	D(\II-C_{u_1}) = \left(
	    \begin{array}{c|c|c}
		\alpha & 0 & 0 \\ \hline
		Q(u_1) & -u_1 & 0
	    \end{array}
	\right),\ 
	[f]_{\mathcal{B}} D^\T = \begin{pmatrix}
	    0 & 1\  \\
	    B_{1,i} & 0\  \\
	    \vdots & \vdots \\
	    B_{2m,i} & 0\  \\
	    0 & 0
	\end{pmatrix},
    \end{equation*}
    where $\alpha = [Bu_1^\T]_i = [u_1B]_i$. Finally, 
    \begin{equation*}
	D(\II-C_{u_1}) [f]_{\mathcal{B}} D^\T = \begin{pmatrix}
	    0 & \alpha \\
	    -\alpha & Q(u_1)
	\end{pmatrix},
    \end{equation*}
    so $\det(D(\II-C_{u_1})FD^\T) = \alpha^2$ as needed. Since the quasideterminant and 
    the spinor norm are multiplicative (Theorems 11.43 and 11.50 in \cite{Taylor}),
    and $\hat{A} \in \Omega_{2m+2}(F,\hat{Q})$, we conclude that $A$ is
    an element of $\Omega_{2m}(F,Q)$ and it follows that the stabiliser
    of $v_0$ in $\Omega_{2m}(F,\hat{Q})$ is indeed a subgroup of shape
    $W\cn\Omega_{2m}(F,Q)$.
    
    Lastly, if we stabilise $v_0$ and $v_1$ simultaneously, a general element in the
    stabiliser takes the form
    \begin{equation*}
	\left(
	    \begin{array}{c|c|c}
		1 & 0 & 0 \\ \hline 
		 & & \\
		0 &\ \ \ \ \ A\ \ \ \ \  & 0 \\ 
		 & & \\ \hline 
		0 & 0 & 1
	    \end{array}
	\right),
    \end{equation*}
    so the stabiliser of $(v_0,v_1)$ is $\Omega_{2m}(F,Q)$. 
\end{proof}

% ------------------------------------------------------------------------

%%% Local Variables: 
%%% mode: latex
%%% TeX-master: "../thesis"
%%% End: 
