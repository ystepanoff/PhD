%%% Thesis Introduction --------------------------------------------------
\chapter*{Introduction}
\addcontentsline{toc}{chapter}{Introduction}

\ifpdf
    \graphicspath{{Introduction/IntroductionFigs/PNG/}{Introduction/IntroductionFigs/PDF/}{Introduction/IntroductionFigs/}}
\else
    \graphicspath{{Introduction/IntroductionFigs/EPS/}{Introduction/IntroductionFigs/}}
\fi
\label{Intro}

One may think of the exceptional groups of Lie type 
as of the wonders of our
world which are still very far from our understanding. Yet these
groups can be found almost everywhere in contemporary mathematics
and physics. Moreover, physicists have some high hopes that
exceptional groups will somehow reveal themselves in the theory
of quantum mechanics. 

The history of the exceptional groups began roughly between
the 19\textsuperscript{th} and 
20\textsuperscript{th} centuries. Elie Cartan (1869--1951)
first completely classified simple Lie algebras, and then by 
determining the real forms of complex algebras he classified the
simple real Lie algebras. A rigorous and highly readable historical
account of this classification can be found in 
\cite{HawkinsThomas}. Apart from the four infinite families
of ``classical'' Lie algebras and Lie groups corresponding to
them, there are five isolated ones, known to the world as
$\G_2$, $\F_4$, $\E_6$, $\E_7$, and $\E_8$.

In this thesis we devote our interest to the construction of the
groups of type $\E_6$. This goes back more than a hundred years
to the works of Leonard Eugene Dickson (1874--1954),
who characterised $\E_6$ as a $27$-dimensional group with an
invariant cubic form \cite{Dickson1, Dickson2}. 
Dickson also managed to write down a 
large number of group generators. He used $27$ coordinates
labelled $x_i$, $y_i$, and $z_{ij} = -z_{ji}$, where 
$i,j \in \{1,2,3,4,5,6\}$ and $i \neq j$. His group is defined
as the stabiliser of a cubic form with $45$ terms
\begin{equation}
	\sum\limits_{i \neq j} x_i y_j z_{ij} + 
	\sum z_{ij} z_{kl} z_{mn},
\end{equation}
where the second sum is taken over all partitions 
$\{\{i, j\}, \{k, l\}, \{m, n\}\}$ of 
the index set
\mbox{$\{1,2,3,4,5,6\}$}, ordered so that $\bigl(\begin{smallmatrix}
  1 & 2 & 3 & 4 & 5 & 6 \\
  i & j & k & l & m & n
\end{smallmatrix}\bigr)$ is 
an even permutation.

In 1955 Claude Chevalley (1909--1984) obtained a uniform
construction of what are now known as Chevalley groups 
\cite{Chevalley}. Chevalley's construction included $\E_6$ albeit
he constructs the $78$-dimensional representation whereas
Dickson obtained the $27$-dimensional representation, which is
the minimal one. 

There was another major breakthrough following Dickson's 
construction of $\E_6$: in $1932$ a new subatomic particle called
neutron was discovered, which indicated the need for a new 
algebraic underpinning of quantum mechanics.
In the same year a German physicist 
Pascual Jordan\footnote{
Naming the first-born son Pascual was a family tradition. 
Furthermore, while in English `Jordan' is pronounced with the
first sound being the same as in the word `judge', in German
the first sound is the same as in `yolk'.
} (1902--1980) introduced Jordan algebras as a tool which was 
supposed to illuminate the behaviour of observable particles 
in quantum mechanics. 

In $1934$, Pascual Jordan, John von Neumann (1903--1957), 
and Eugene Wigner (1902--1995) introduced Jordan 
algebras and octonions into physics \cite{JordanNeumannWigner}. 
Although their attempt to formulate a new quantum mechanics was
unsuccessful, Jordan algebras turned out to have astonishing 
connections with many areas of mathematics. A by-product of
physical investigation was the discovery of the $27$-dimensional
exceptional Jordan algebra also known as the Albert algebra, named 
after Abraham Adrian Albert (1905--1972).

Although abandoned by physicists, the Albert algebra turned out
to be of high interest to mathematicians. This $27$-dimensional
algebra consists of $3\times 3$ Hermitian matrices written 
over octonions, with the multiplication given by
\begin{equation}
X\circ Y = \frac12 (XY + YX).	
\end{equation}
Hans Freudenthal (1905--1990) showed that the stabiliser
of a certain cubic form on this $27$-dimensional space is a 
group of type $\E_6$ \cite{Freudenthal}. 
George Seligman (born 1927) proved that the 
automorphism group of a split Albert algebra over any field
$F$ is isomorphic to the Chevalley group $\F_4(F)$ 
\cite{Seligman}. Nathan Jacobson (1910--1999), 
inspired by the works of Dickson and
Chevalley, studied the automorphism 
group of the
Albert algebra, and the stabiliser of the determinant over the
fields of characteristic not $2$ or $3$ in the series of papers
\cite{JacobsonOne,JacobsonTwo,JacobsonThree}. For instance, he 
proved that if an Albert algebra contains nilpotent elements, 
then the automorphism group is simple. It must have been 
implicit that the
determinant of the elements in the Albert algebra is essentially
the same as Dickson's cubic form, although Jacobson does not refer
to Dickson. Moreover, although cases of characteristic $2$ and $3$
were of no problem to Dickson, they were still problematic in
Jacobson's construction.

Michael Aschbacher (born 1944) also addresses the 
construction of the 
groups of type $\E_6$ without mentioning Albert
algebra or octonions at all. Though of fundamental importance,
Aschbacher's construction is of rather abstract nature and some
of the results need to be investigated in detail. 
Thus, some of the structural questions require further research. 

In his famous book, `The Finite Simple Groups' \cite{WilsonBook},
and also in subsequent preprint \cite{WilsonPaper},
Robert Arnott Wilson (born 1958) sketches the
construction of the finite simple groups $\F_4(q)$, 
$\E_6(q)$, and ${}^2\E_6(q)$. The purpose of this thesis is,
having this sketch as a basis, to fill in the major gaps and 
provide a complete and self-contained construction of the groups 
of type $\E_6$, and also investigate the possibility to adopt this
approach to the groups of type ${}^2\E_6$ over an arbitrary field.

In the late 1980s the problem of classifying maximal subgroups 
came into prominence. One of the most notable examples is
Kay Magaard's (1962--2018) unpublished thesis \cite{Magaard}, which deals
with the maximal subgroups of finite simple groups $\F_4(q)$
where the characteristic is not $2$ or $3$. The series of papers by 
Aschbacher \cite{Asch1,Asch2,Asch3,Asch4} addresses the investigation
of the maximal subgroups of $\E_6$. It turns out that the $27$-dimensional
representation of the generic cover reveals much more structure rather 
than the standard $78$-dimensional representation on the Lie algebra.
However, Aschbacher does not provide a complete list of maximal subgroups, 
which indicates a need for a concrete and easy-to-use construction, which can
help to understand the subgroup structure better. 

%%% ----------------------------------------------------------------------


%%% Local Variables: 
%%% mode: latex
%%% TeX-master: "../thesis"
%%% End: 
