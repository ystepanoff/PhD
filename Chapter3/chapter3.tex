\chapter{Groups of type ${}^2\EE_6$}
\ifpdf
    \graphicspath{{Chapter3/Chapter3Figs/PNG/}{Chapter3/Chapter3Figs/PDF/}{Chapter3/Chapter3Figs/}}
\else
    \graphicspath{{Chapter3/Chapter3Figs/EPS/}{Chapter3/Chapter3Figs/}}
\fi
%leave these commands here

\section{Quadratic field extensions}

Let $F$ and $K$ be two fields such that $F$ is a subfield of $K$. We say that 
$K$ is an \textit{extension field} of $F$. The \textit{degree} of $K$ over $F$, 
denoted by $[K:F]$, is the dimension of $K$ as a vector space over $F$. We denote the
extension of $F$ by $K$ as $K : F$.

A non-zero polynomial $f \in F[x]$ is called \textit{separable}, if each root of $f$ has 
multiplicity $1$. If $\alpha \in F$ is algebraic, that is, $\alpha$ is a root of some polynomial
$g \in F[x]$, then $\alpha$ is called \textit{separable}, if its minimal polynomial is separable. 

We have defined what it means for a field element and a polynomial to be separable. Suppose that
$[K:F]$ is finite, then $K : F$ is a \textit{separable} extension, if every element of $K$ is 
separable over $F$. We also say that $K : F$ is \textit{normal}, if every irreducible 
polynomial $f \in F[x]$ that has a root in K, splits into linear factors in $K[x]$. An extension
$K : F$ which is normal, separable, and of finite degree, is called a \textit{Galois extension}.
We are not going into too much detail here, for there are various  well-known references on theory
of field extensions, of very high quality, for instance, \cite{PeterCameron}, \cite{DummitFoote}, \cite{Lang}, or \cite{Stewart}. In this chapter
we are interested in Galois extensions of degree $2$. 
In case when $F = \Fq$, we have $|K| = q^2$, and so $K = \Fqs$ (see, for example, \cite{Moore}).

Given an extension $K : F$, we define the \textit{Galois group} of $K$ over $F$, denoted
$\Gal(K : F)$, to be the group
of automorphisms of $K$ that fix $F$ elementwise. In other words,
\begin{equation}
	\Gal(K : F) = \{\ \sigma \in \Aut(K)\ \big|\ \alpha^{\sigma} = \alpha\ \mbox{for all}\ 
		\alpha \in F\ \}.
\end{equation}
The most important result for us here is the following theorem.
\begin{theorem}
	Let $K : F$ be a Galois extension. Then
	\begin{equation}
		| \Gal(K : F) | = [K : F].
	\end{equation}
\end{theorem}
It follows that in case $[K:F] = 2$ there is a unique non-trivial field automorphism 
$\sigma : K \rightarrow K$, fixing $F$ elementwise. If $F = \Fq$ and $K = \Fqs$, then
$\sigma$ takes the form $\sigma : \lambda \mapsto \lambda^q$. 

% Moore: Doubly-infinite systems of simple groups

\section{Spaces with two forms}

Let $K : F$ be a Galois extension of degree $2$, 
and let $s$ be a $K$-automorphism
with $F$ being its fixed field. Let also $V$ be a $2m$-dimensional vector space over $K$ with 
a quadratic form $Q$ and a conjugate-symmetric sesquilinear form 
\mbox{$B : V\times V \rightarrow K$},
defined with respect to $\sigma$. Denote by $f$ the polar form of $Q$, and suppose that 
\mbox{$\mathcal{B} = \{ e_1, f_1, e_2, f_2, ..., e_m, f_m \}$}
is a hyperbolic basis of $V$ with respect to $f$, i.e.
\begin{equation}
	Q(e_i) = Q(f_i) = 0,\ f(e_i, f_i) = 1,
\end{equation}
and $f(e_i,e_j) = f(e_i,f_j) = 0$ for $i \neq j$. The above said means that $(V,Q)$ is a hyperbolic
orthogonal space. Denote by $G$ the maximal amongst all the subgroups of 
$\GO(V, Q)$ which preserve $B$. If $U$ 
is a subspace of $V$, then we denote the restrictions of $f$ and $B$ on $U$ as $f_U$ and $B_U$ 
respectively. We say that an element $v \in V$ is \textit{singular isotropic}, if $Q(v) = B(v,v) = 0$. 

\begin{definition}
	A $(Q,B)$-subspace of $V$ is an $F$-subspace $U$ of $V$ such that \mbox{$V = U \otimes_F K$}, 
	$f_U = B_U$ is an $F$-form on $U$, and $Q_U$ is a non-degenerate quadratic form on $U$ of 
	Witt index at least $2$. 
\end{definition}

\begin{proposition}
	\label{prop:3_2forms}
	If $U$ is a $(Q,B)$-subspace of $V$, then it is the unique $(Q,B)$-subspace of $V$, and
	$G = \GO(U,Q)$. 
\end{proposition}

\begin{proposition}
	\label{prop:3_2forms_orbits}
	Let $U$ be a $(Q,B)$-subspace of $V$ if Witt index at least $2$. Fix $\lambda \in K 
	\setminus F$. The group $G$ has two orbits on doubly singular points with representatives
	$\langle u \rangle$ and $\langle u + \lambda v \rangle$, where $u,v \in U$ and 
	$\langle u, v \rangle$ 	is a singular line. 
\end{proposition}

\section{Hermitean form in $\J$ and the group ${}^2\SE_6^K(F)$}

Suppose as before $K : F$ is a quadratic Galois extension with $F$ being a fixed field of a
$K$-automorphism $\sigma$. In this chapter $\O_F$ will always be a split octonion algebra 
over $F$ and $\O_K = \O_F \otimes_F K$. We also use the same basis 
$\{\ e_i\ \mid\ i \in \pm I\ \}$ as in Section \ref{section:split_basis}.

We slightly abuse the notation here denoting by $\s$ also the automorphism of $\O_K$ induced by 
the field automorphism $\sigma$:
\begin{equation}
	\left( \sum_{i\in \pm I} \lambda_i e_i \right)^{\s} = 
	\sum_{i\in \pm I} \lambda_{i}^{\sigma} e_i. 
\end{equation}
This, however, should not create any difficulties for the reader. 
Consider the following Hermitean form defined on the elements of $\O_K$:
\begin{equation}
	h(x) = A \ovA^{\s} + A^{\s} \ovA = \Tr(A \ovA^{\s}).
\end{equation}
On the Albert space $\J = \J_K$ this induces the Hermitean form $\H$, where
\begin{equation}
	\H( (a,b,c\mid A,B,C) ) = 
		a a^{\s} + b b^{\s} + c c^{\s} + 
		\Tr(A \ovA^{\s} + B\ovB^{\s} + C\ovC^{\s}).
\end{equation}
As usual, the sesquilinear form associated with $\H$ is obtained by polarising the Hermitean form:
\begin{equation}
	\inner{X}{Y}_{\H} = H(X+Y) - H(X) - H(Y).
\end{equation}

Using the construction from previous chapter, we obtain the group $\SE_6(K)$ in the usual 
way. Now define the group ${}^2\SE_6^K(F)$ as the subgroup of $\SE_6(K)$ which preserves 
$\H$. In case $F = \Fq$ and $K = \Fqs$, we denote this by ${}^2\SE_6(q)$. As before,
the group ${}^2\E_6(F/K)$ is defined as the quotient of ${}^2\SE_6^K(F)$ by its centre. 

\section{Some elements of ${}^2\SE_6^K(F)$}

Let $X = (a,b,c\mid A,B,C)$ be an arbitrary element of $\J = \J_K$. First of all we notice that 
the matrices $\delta$ and $\tau$ preserve the Hermitean form, so they encode the elements of 
${}^2\SE_6^K(F)$.

The elements $P_u$ with $u$ written over the small field and 
such that $\NN(u) = 1$ are also of interest. 

\begin{lemma}
	\label{lemma:3_pu_hermitean}
	The actions on $\J$ of the elements $P_u = \diag(u,\ovu,1)$ such that $u \in \O_F$ and $\NN(u) = 1$,
	preserve $\H$.
\end{lemma}

\begin{proof}
	Recall that the action on $\J$ is given by
	\begin{equation*}
		\begin{array}{r@{\;}c@{\;}l}
			P_u : (a, b, c \mid A, B, C) & \mapsto & (a, b, c \mid u A, B u, \ovu C \ovu),
		\end{array}
	\end{equation*}
	so the individual terms are being mapped in the following way:
	\begin{equation*}
		\begin{array}{r@{\;}c@{\;}l}
			a a^{\s} & \mapsto & a a^{\s}, \\
			b b^{\s} & \mapsto & b b^{\s}, \\
			c c^{\s} & \mapsto & c c^{\s}, \\
			\Tr( A \ovA^{\s} ) & \mapsto & \Tr( (u A) (\ovA^{\s} \ovu) ), \\
			\Tr( B \ovB^{\s} ) & \mapsto & \Tr( (B u) (\ovu \ovB^{\s} ), \\
			\Tr( \ovu C \ovu ) & \mapsto & \Tr( (\ovu C \ovu) (u \ovC^{\s} u) ).
		\end{array}
	\end{equation*}
	Note that $u^{\s} = u$ since $u \in \O_F$. We find
	\begin{multline*}
		\Tr( (u A) (\ovA^{\s} \ovu) ) = \Tr( ((u A) \ovA^{\s}) \ovu ) = 
		\Tr( \ovu ((u A) \ovA^{\s}) ) \\
		= \Tr( ( \ovu (u A) ) \ovA^{\s} ) = 
		\Tr( ( (\ovu u) A ) \ovA^{\s} ) = \Tr(A \ovA^{\s}).
	\end{multline*}
	Similarly,
	\begin{equation*}
		 \Tr( (B u) (\ovu \ovB^{\s} ) = \Tr( ((B u) \ovu ) \ovB^{\s} ) = 
		 \Tr( (B (u \ovu) ) \ovB^{\s} ) = \Tr(B \ovB^{\s}).
	\end{equation*}
	For the last term we have
	\begin{equation*}
		\Tr( (\ovu C \ovu) (u \ovC^{\s} u) ) = \inner{\ovu C \ovu}{\ovu C^{\s} \ovu}.
	\end{equation*}
	Now, as we know from Lemma \ref{lemma:1_spin8plus}, the map $x \mapsto \ovu x \ovu$ is
	a product of two reflexions, hence,  
	\begin{equation*}
		\inner{\ovu C \ovu}{\ovu C^{\s} \ovu} = \inner{C}{C^{\s}} = \Tr(C\ovC^{\s}).
	\end{equation*}
\end{proof}

\begin{lemma}
	\label{lemma:3_isotropic}
	Let $x \in \O_K$ be such that $\ovx^{\s} x = x \ovx^{\s} = 0$. Then $x\ovx = 0$.
\end{lemma}

\begin{proof}
	If $x = 0$, then the result is trivial. Assume $x$ is non-zero, $\ovx^{\s} x = x\ovx^{\s} = 0$.
	If $x \ovx \neq 0$, then $x$ is invertible, and so $\ovx^{\s} = 0$, which implies $x = 0$, a 
	contradiction.
\end{proof}

Consider the matrices
\begin{equation}
	N_x = \begin{bmatrix}
		1 & x & 0 \\
		-\ovx^{\s} & 1 & 0 \\
		0 & 0 & 1
	\end{bmatrix},\quad
	N_x' = \begin{bmatrix}
		1 & 0 & 0 \\
		0 & 1 & x \\
		0 & -\ovx^{\s} & 1
	\end{bmatrix},\quad
	N_x'' = \begin{bmatrix}
		1 & 0 & -\ovx^{\s} \\
		0 & 1 & 0 \\
		x & 0 & 1
	\end{bmatrix},
\end{equation}
where $x\ovx^{\s} = \ovx^{\s}x = 0$ and $x,\ovx^{\s}$ generate a sociable subalgebra. 
It is easy to see that these encode the elements of $\SE_6(K)$. Indeed,
\begin{equation}
	\begin{bmatrix}
		1 & x \\
		0 & 1
	\end{bmatrix}
	\begin{bmatrix}
		1 & 0 \\
		-\ovx ^{\s} & 1
	\end{bmatrix} = 
	\begin{bmatrix}
		1 - x\ovx ^{\s} & x \\
		-\ovx ^{\s} & 1 
	\end{bmatrix} = 
	\begin{bmatrix}
		1 & x \\
		-\ovx ^{\s} & 1
	\end{bmatrix}.
\end{equation}
So, the elements $N_x$, $N_x'$, and $N_x''$ preserve the Dickson--Freudenthal determinant. 
To verify that they preserve $\H$, we look at the action on $\J$:
\begin{equation}
	\label{eq:3_nx_image}
	\begin{array}{r@{\;}c@{\;}l}
		N_x: (a,b,c\mid A,B,C) & \mapsto &
		(a-\Tr(C\ovx ^{\s}), b+\Tr(\ovC x), c \mid \\
		& & \mid	A+\ovx \ovB , B-\ovA \ovx ^{\s},C-x^{\s}\ovC x+ax-bx^{\s}), \\
		
		N_x': (a,b,c\mid A,B,C) & \mapsto & 
		(a, b-\Tr(A \ovx^{\s}), c+\Tr(\ovA x) \mid \\
		& & \mid	A - x^{\s} \ovA x + bx-cx^{\s}, B + \ovx \ovC, C - \ovB \ovx^{\s}), \\
		
		N_x'': (a,b,c\mid A,B,C) & \mapsto &
		(a+\Tr(\ovB x), b, c-\Tr(B\ovx^{\s}) \mid \\
		& & \mid 	A-\ovC\ovx^{\s}, B-x^{\s}\ovB x + cx - ax^{\s}, C+\ovx\ovA).
	\end{array}
\end{equation}
We need to prove an auxiliary lemma.

\begin{lemma}
	\label{lemma:3_oct_aux}
	Suppose that $x,y,z \in \O_F$ with $x \ovx  = 0$. Then
		\begin{enumerate}[(i)]
			\item $x\Tr(yx) = x(yx)$,
			\item $\Tr( (xy)(z\ovx ) ) = 0$.		
		\end{enumerate}
\end{lemma}

\begin{proof}
	\leavevmode
	\begin{enumerate}[(i)]
	
	\item $x\Tr(yx) = x (yx + \ovx \ovy ) = x(yx) + x(\ovx \ovy )
		 = x(yx) + (x\ovx )y = x(yx)$,
		 
	\item $\Tr( (xy)(z\ovx ) ) = \Tr( (z\ovx ) (xy) ) = \Tr( ((z\ovx )x) y ) = 
		\Tr(z(x\ovx )y) = 0$. \qedhere
	\end{enumerate}
\end{proof}

Obviously, it is enough to verify that the elements $N_x$ preserve the Hermitean form $\H$. 
The individual terms in $\H(X) = aa^{\s} + bb^{\s} + cc^{\s} + \Tr(A\ovA ^{\s} + B\ovB ^{\s} +
C\ovC ^{\s})$ are being mapped in the following way:
\begin{equation}
	\begin{array}{r@{\;}c@{\;}l}
		aa^{\s} & \mapsto & aa^{\s} - a^{\s}\Tr(C\ovx ^{\s}) - a\Tr(C^{\s} \ovx ) 
					+ \Tr(C^{\s}\ovx ) \Tr(C\ovx ^{\s}), \\
		bb^{\s} & \mapsto & bb^{\s} + b^{\s}\Tr(\ovC x) + b\Tr(\ovC ^{\s}x^{\s}) +
					\Tr(\ovC x)\Tr(\ovC ^{\s}x^{\s}), \\
		cc^{\s} & \mapsto & cc^{\s}, \\
		\Tr(A\ovA ^{\s}) & \mapsto & \Tr(A\ovA ^{\s}) + \Tr(AB^{\s}x^{\s}) + 
					\Tr(\ovx \ovB \ovA ^{\s}) + 
					\Tr((\ovx \ovB )(B^{\s}x^{\s})), \\
		\Tr(B\ovB ^{\s}) & \mapsto & \Tr(B\ovB ^{\s}) - 
					\Tr(\ovA \ovx ^{\s} \ovB ^{\s}) - \Tr(BxA^{\s}) + 
					\Tr((\ovA \ovx ^{\s})(xA^{\s})), \\
		\Tr(C\ovC ^{\s}) & \mapsto & \Tr(C\ovC ^{\s}) - 
					\Tr(C(\ovx ^{\s}C^{\s}\ovx )) - \Tr(C^{\s}(\ovx C\ovx ^{\s}))
					+ a\Tr(x\ovC ^{\s}) + \\
					& & + a^{\s}\Tr(C\ovx ^{\s}) -
					b^{\s}\Tr(C\ovx ) - b\Tr(x^{\s}\ovC ^{\s}).
	\end{array}
\end{equation}
Using Lemmas \ref{lemma:3_isotropic} and
\ref{lemma:3_oct_aux}, we get
\mbox{$\Tr((\ovA \ovx ^{\s})(xA^{\s})) = 0= \Tr((\ovx \ovB )(B^{\s}x^{\s}))$}.
Next, we also obtain $\Tr(C( \ovx ^{\s} C^{\s} \ovx )) = \Tr(C \ovx ^{\s}
\Tr(C^{\sigma} \ovx )) = \Tr(C^{\sigma} \ovx )\Tr(C\ovx ^{\s})$. Likewise,
we get $\Tr(C^{\s}(\ovx C\ovx ^{\s})) = \Tr ((x^{\s}\ovC x)\ovC ^{\s}) = 
\Tr(\ovC ^{\s} (x^{\s}\ovC x)) = \Tr(\ovC ^{\s} x^{\s})\Tr(\ovC x)$. We see
that all the terms except $aa^{\s},bb^{\s},cc^{\s}$ and $\Tr(A\ovA ^{\s}),
\Tr(B\ovB ^{\s}), \Tr(C\ovC ^{\s})$ cancel out, so it follows that the elements
$N_x$ preserve the Hermitean form. Hence, we have shown the following. 

\begin{proposition}
	The matrices $N_x$, $N_x'$, and $N_x''$ where $x\ovx^{\s} = 0 = \ovx^{\s}x$ and $x, \ovx^{\s}$
	generate a sociable subalgebra, encode the elements of ${}^2\SE_6^K(F)$. 
\end{proposition}

\begin{lemma}
	\label{lemma:sociable}
	If $x \in \O$ is such that $x \ovx^{\s} = 0 = \ovx^{\s} x$, with $x,\ovx^{\s}$ generating
	a sociable subalgebra, then $x^{\s} y x = x y x^{\s}$ for all $y \in \O_K$.  
\end{lemma}

Of great interest for us is the action of $N_x$ on $\J_{10}^{abC}$. The rest of the section
is devoted to proving the following result.

\begin{theorem}
	\label{theorem:3_nx_omega}
	The actions of the elements $N_x$ on $\J_{10}^{abC}$, where $x$ is such that 
	$x \ovx^{\s} = 0 = \ovx^{\s} x$, with $x,\ovx^{\s}$ generating a sociable subalgebra, 
	generate a group of type $\Omega_{10}^{-,K}(F)$, as $x$ ranges through all suitable octonions
	in $\O_K$.  
\end{theorem}

We prove this theorem in the series of steps. First, consider the $4$-dimensional 
$K$-subspace $V_4$ of $\J_K$, 
spanned by the Albert vectors of the form $(a,b,0\mid 0,0,C)$ with 
$C \in \langle e_{-1}, e_{1} \rangle$. 

\begin{lemma}
	The actions of the elements $N_{\lambda e_{ \pm 1 }}$ on $V_4$, where $\lambda \in K$, 
	generate a group of type $\Omega_4^{-,K}(F)$. 
\end{lemma}

\begin{proof}
	Consider the basis $\mathcal{B}$ of $V_4$, consisting of the vectors $v_3$, $v_2$, $v_1$, and $v_4$, in that order, where
	\begin{equation*}
		\begin{array}{r@{\;}c@{\;}l}
			v_1 & = & (1,0,0\mid 0,0,0), \\
			v_2 & = & (0,1,0\mid 0,0,0), \\
			v_3 & = & (0,0,0\mid 0,0,e_{-1}), \\
			v_4 & = & (0,0,0\mid 0,0,e_1).
		\end{array}
	\end{equation*}
	
	With respect to $\mathcal{B}$ the $4 \times 4$ matrices
	$[N_{\lambda e_{-1}}]_{\mathcal{B}}$ and  $[N_{\lambda e_1}]_{\mathcal{B}}$ take the form
	\begin{equation*}
		[N_{\lambda e_{-1}}]_{\mathcal{B}}	= \begin{bmatrix}
			1 & 0 & 0 & 0 \\
			-\lambda^{\s} & 1 & 0 & 0 \\
			\lambda & 0 & 1 & 0 \\
			-\lambda\lambda^{\s} & \lambda & -\lambda^{\s} & 1
		\end{bmatrix},\quad 
		[N_{\lambda e_1}]_{\mathcal{B}} = \begin{bmatrix}
			1 & \phantom{-}\lambda & -\lambda^{\s} & -\lambda\lambda^{\s} \\
			0 & \phantom{-}1 & 0 & -\lambda^{\s} \\
			0 & \phantom{-}0 & 1 & \lambda \\
			0 & \phantom{-}0 & 0 & 1
		\end{bmatrix}.
	\end{equation*}
	We notice that 
	\begin{equation*}
		[N_{\lambda e_{-1}}]_{\mathcal{B}} = \begin{bmatrix}
			1 & 0 \\
			\lambda & 1 
		\end{bmatrix} \otimes
		\begin{bmatrix}
			1 & 0 \\
			-\lambda^{\s} & 1
		\end{bmatrix},\quad
		[N_{\lambda e_1}]_{\mathcal{B}} = \begin{bmatrix}
			1 & -\lambda^{\s} \\
			0 & 1
		\end{bmatrix} \otimes
		\begin{bmatrix}
			1 & \lambda \\
			0 & 1
		\end{bmatrix},
	\end{equation*}
	where $\otimes$ is the Kronecker product of two matrices. The mapping
	\begin{equation*}
		\begin{bmatrix}
			1 & 0 \\
			\lambda & 1 
		\end{bmatrix} \mapsto
		\begin{bmatrix}
			1 & 0 \\
			\lambda & 1
		\end{bmatrix} \otimes
		\begin{bmatrix}
			1 & 0 \\
			-\lambda^{\s} & 1
		\end{bmatrix},\quad 
		\begin{bmatrix}
			1 & \lambda \\
			0 & 1
		\end{bmatrix} \mapsto
		\begin{bmatrix}
			1 & -\lambda^{\s} \\
			0 & 1
		\end{bmatrix} \otimes
		\begin{bmatrix}
			1 & \lambda \\
			0 & 1
		\end{bmatrix}
	\end{equation*}
	can be extended to a homomorphism $\phi$ which is obviously surjective as $\lambda$
	 ranges
	through the whole field $K$. Its kernel is a subgroup
	\begin{equation*}
		\ker(\phi) = 
		\left\langle
			\begin{bmatrix}
				-1 & 0 \\
				0 & -1
			\end{bmatrix}
		\right\rangle,
	\end{equation*}
	which has order $2$, so we get the action of the group $\PSL_2(K)$ on $V_4$ since
	the matrices
	\begin{equation*}
		\begin{bmatrix}
			1 & 0 \\
			\lambda & 1
		\end{bmatrix},\quad
		\begin{bmatrix}
			1 & \lambda \\
			0 & 1 
		\end{bmatrix}
	\end{equation*}
	generate a group $\SL_2(K)$ by a well-known result.
	Therefore, as $\PSL_2(K) \cong \Omega_4^{-,K}(F)$ (see, for example, \S 7 in \cite{Waerden}), 
	we have the action of $\Omega_4^{-,K}(F)$. 
\end{proof}

Consider now a $6$-dimensional $K$-space $V_6$ with the basis $\mathcal{B}$ consisting of the vectors
$v_5, v_3, v_2, v_1, v_4, v_6$, in that order, where
\begin{equation}
	\begin{array}{r@{\;}c@{\;}l}
		v_5 & = & (0,0,0\mid 0,0,e_{\bar{\omega}}), \\
		v_6 & = & (0,0,0\mid 0,0,e_{-\bar{\omega}}). \\
	\end{array} 
\end{equation}
We are interested in the group, obtained by adjoining the action of $N_{\mu e_{\bar{\omega}}}$ to 
$\Omega_4^{-,K}(F)$. With respect to the chosen basis, this new element has the following matrix form:
\begin{equation}
	\begin{bmatrix}[c]
		1 & 0 &  \phantom{-}0 & 0 & 0 & \phantom{-}0 \\
		0 & 1 & \phantom{-}0 & 0 & 0 & \phantom{-}0 \\
		-\mu^{\s} & 0 & \phantom{-}1 & 0 & 0 & \phantom{-}0 \\
		\mu & 0 & \phantom{-}0 & 1 & 0 & \phantom{-}0 \\
		0 & 0 & \phantom{-}0 & 0 & 1 & \phantom{-}0 \\
		-\mu \mu^{\s} & 0 & \phantom{-}\mu & -\mu^{\s} & 0 & \phantom{-}1
	\end{bmatrix}.
\end{equation}
We are interested in the vector $u = (0,\mu, -\mu^{\s}, 0)$. Recall that with respect to the basis
$v_3, v_2, v_1, v_4$ the elements $N_{\lambda e_{-1}}$ and $N_{\lambda e_1}$ are represented 
by the following matrices respectively:
\begin{equation}
		 \begin{bmatrix}
			1 & 0 & 0 & 0 \\
			-\lambda^{\s} & 1 & 0 & 0 \\
			\lambda & 0 & 1 & 0 \\
			-\lambda\lambda^{\s} & \lambda & -\lambda^{\s} & 1
		\end{bmatrix},\quad 
		\begin{bmatrix}
			1 & \phantom{-}\lambda & -\lambda^{\s} & -\lambda\lambda^{\s} \\
			0 & \phantom{-}1 & 0 & -\lambda^{\s} \\
			0 & \phantom{-}0 & 1 & \lambda \\
			0 & \phantom{-}0 & 0 & 1
		\end{bmatrix}.	
\end{equation}
The element $N_{\lambda e_{-1}}$, represented by the lower triangular matrix, sends our vector
$u$
to $(-\lambda^{\s} \mu - \lambda\mu^{\s}, \mu, -\mu^{\s}, 0)$, while 
$N_{\lambda e_1}$, represented by the upper triangular matrix, sends it to
$(0,\mu, -\mu^{\s}, \lambda\mu + \lambda^{\s}\mu^{\s})$. Note that
$\lambda^{s}\mu + \lambda\mu^{\s}$ and $\lambda\mu + \lambda^{\s}\mu^{\s}$ are
the elements of $F$, and the pair $(\mu, -\mu^{\s})$ generates $K \cong F^{2}$ as
$\mu$ ranges through $K$. Hence, we have the action of $\Omega_4^{-,K}(F)$ on
$F^4$, so by adjoining the element $N_{\mu e_{\bar{\omega}}}$ to $\Omega_4^{-,K}(F)$
we have obtained a group of shape $F^{4}\cn\Omega_4^{-,K}(F)$. 

We again use the results of Appendix A. Consider the $8$-space $V_8$ spanned by the Albert vectors $(a,b,0\mid 0,0,C)$ with 
$C \in \langle e_{-1}, e_{\bar{\omega}}, e_{\omega}, 
e_{- \omega}, e_{-\bar{\omega}}, e_1 \rangle$, and consider two isotropic vectors
\begin{equation}
	\begin{array}{r@{\;}c@{\;}l}
		u_{\omega} & = & (0,0,0 \mid 0,0,e_{\omega}), \\
		u_{-\omega} & = & (0,0,0 \mid 0,0,e_{-\omega}),
	\end{array}
\end{equation}
which are fixed by our copy of $\Omega_6^{-,K}(F)$. The action of $N_{e_{\omega}}$ on $V_8$
preserves $u_{\omega}$ but not $u_{-\omega}$, and therefore adjoining this element to 
$\Omega_6^{-,K}(F)$ we get the action of the group $V_6\cn\Omega_6^{-,K}(F)$. The element
$N_{e_{-\omega}}$ does not preserve the $1$-space $\langle u_{\omega} \rangle$, so appending it to 
$V_6\cn\Omega_6^{-,K}(F)$, we get the action of $\Omega_8^{-,K}(F)$ on $V_8$. 

Finally, we choose two isotropic Albert vectors
\begin{equation}
	\begin{array}{r@{\;}c@{\;}l}
		u_0 & = & (0,0,0\mid 0,0,e_0), \\
		u_{-0} & = & (0,0,0\mid 0,0,e_{-0}).
	\end{array}
\end{equation}
in $\J_{10}^{abC}$. We adjoin the element $N_{e_{0}}$ which fixes $u_0$ but not $u_{-0}$ to 
get the action of the group $V_8\cn \Omega_8^{-,K}(F)$. Appending to this the action of
$N_{e_{-0}}$ which does not preserve $\langle u_0 \rangle$, we obtain the action of
$\Omega_{10}^{-,K}(F)$.

\section{Action of ${}^2\SE_6^K(F)$ on white points}

As in the case of $\SE_6$, we are interested in the action on white points. We will, however, see 
that although $SE_6$ acts transitively on white points, the action of ${}^2\SE_6^K(F)$ splits
into several orbits. We say that a non-zero Albert vector $X$ is \textit{isotropic}, if
$\H(X) = 0$. 

We first consider some examples. Suppose $X_1 = (0,0,0\mid 0,0,e_0)$. As we know (\ref{eq:17_space}), 
it determines 
a $17$-space $	\{\ (a,b,0 \mid A,B,C)\ \big|\ e_0 A = B e_0 = \Tr(e_0 \ovC) = 0\ \}$. A
straightforward calculation shows that
this $17$-space $U_1$ is spanned by the Albert vectors of the form \mbox{$(a,b,0\mid A,B,C)$} with
\begin{equation}
	\begin{array}{r@{\;}c@{\;}l}
		A & \in & \langle e_{\bar{\omega}}, e_{\omega}, e_{-0}, e_1 \rangle, \\
		B & \in & \langle e_{-1}, e_{-0}, e_{-\omega}, e_{-\bar{\omega}} \rangle, \\
		C & \in & \langle e_{-1}, e_{\bar{\omega}}, e_{\omega}, e_0, e_{-\omega}, e_{-\bar{\omega}}, e_1
		\rangle.
	\end{array}
\end{equation}
We are also interested in the radical $R_1$ of $\H$ inside this $17$-space. In our case it is spanned
by the vectors of the form \mbox{$(0,0,0\mid A,B,C)$} with
\begin{equation}
	\begin{array}{r@{\;}c@{\;}l}
		A & \in & \langle e_{\bar{\omega}}, e_{\omega}, e_{-0}, e_1 \rangle, \\
		B & \in & \langle e_{-1}, e_{-0}, e_{-\omega}, e_{-\bar{\omega}} \rangle, \\
		C & \in & \langle e_0 \rangle.
	\end{array}
\end{equation}
In other words, our vector $X_1$ determines the $17$-space $U_1$ and the $9$-dimensional radical $R_1$ 
of $\H$ in $U_1$. Note that $X_1$ is isotropic with respect to $\H$, and also $X_1 \in R_1$. 

It turns out that there is another type of isotropic white vectors. Consider 
$X_2 = (0,0,0 \mid 0,0,e_0 + \lambda e_1)$, where $\lambda \in K\setminus F$. It again determines a 
$17$-space $U_2$, spanned by the Albert vectors of the form $(a,b,0\mid A,B,C)$ with 
\begin{equation}
	\begin{array}{r@{\;}c@{\;}l}
		A & \in & \langle e_{\bar{\omega}}+\lambda e_{-\omega}, e_{\omega}-\lambda e_{-\bar{\omega}}
		, e_{-0}, e_1
		\rangle, \\
		B & \in & \langle e_{-1}+\lambda e_0, e_{-0} - \lambda e_1,
	e_{-\omega}, e_{-\bar{\omega}}  \rangle, \\
		C & \in & \langle e_{-1}-\lambda^2 e_1 - \lambda \cdot 1_{\O}, e_{\bar{\omega}}, e_{\omega},
	e_0 + \lambda e_1, e_{-\omega}, e_{-\bar{\omega}}, e_1 \rangle.
	\end{array}
\end{equation}
We find that the radical $R_2$ of $\H$ in $U_2$ is spanned by the vectors of the form 
$(0,0,0\mid A,B,C)$ with 
\begin{equation}
	\begin{array}{r@{\;}c@{\;}l}
		A & \in & \langle e_{-0}, e_1 \rangle, \\
		B & \in & \langle e_{-\omega}, e_{-\bar{\omega}} \rangle, \\
		C & \in & \langle e_0 + \lambda^{\s} e_1 \rangle,
	\end{array}
\end{equation}
i.e. it is $5$-dimensional. We notice that $X_2$ is isotropic, but in this case $X_2 \not\in R_2$.

We conclude that the white points $\langle X_1 \rangle$ and $\langle X_2 \rangle$ belong to different
orbits under the action of ${}^2\SE_6^K(F)$. Of course, there is also at least one orbit on the 
non-isotropic white points. 

\subsection{Orbits of ${}^2\SE_6^K(F)$ on white points}

\subsection{The stabiliser of type 3 vector}

We are now interested in the stabiliser in ${}^2\SE_6^K(F)$ of 
$X_3 = (0,0,1\mid 0,0,0)$, which is non-isotropic. As we know, the elements
$N_x$ with $x\ovx^{\s} = \ovx^{\s}x = 0$ preserve $X_3$ (\ref{eq:3_nx_image}).
We now prove the following theorem.

\begin{theorem}
	The stabiliser in ${}^2\SE_6^K(F)$ of $X_3 = (0,0,1\mid 0,0,0)$ 
	is a subgroup of shape $\Spin_{10}^{-,K}(F)$. 
\end{theorem}

\begin{proof}
	From Theorem \ref{theorem:3_nx_omega} we know that the actions on $\J_{10}^{abC}$
	of the elements $N_x$ generate $\Omega_{10}^{-,K}(F)$. We use 
	Lemma \ref{lemma:1_2_actions} to conclude that the action on $\J$ is that of
	$\Spin_{10}^{-,K}(F)$. Our aim is to show that this group is the whole stabiliser
	in ${}^2\SE_6^K(F)$ of $X_3$.
	
	Let $G$ be the stabiliser of $X_3$ in ${}^2\SE_6^K(F)$. In particular, $G$ is 
	a subgroup of $\SE_6(K)$, so it preserves the Dickson--Freudenthal determinant
	$\fdet$, and hence it stabilises the $17$-space $U_3 = \J_{17}^{cAB}$ 
	determined by $X_3$. Next, $G$
	is a subgroup of ${}^2\SE_6^K(F)$, so it preserves the Hermitean form $\H$,
	and so $G$ stabilises $U_3^{\perp}$, where $\perp$ is taken with respect to the
	sesquilinear form. 
	Now, $U_3^{\perp}$ has two forms defined on it, so we use Aschbacher's result 
	(Proposition \ref{prop:3_2forms}) to conclude that there exists a $(Q, \H)$-subspace.	
	
	Consider the $10$-dimensional $F$-subspace $V_{10}^-$ spanned by the vectors 
	of the form $(\lambda\cdot 1, -\lambda^{\sigma}\cdot 1, 
	0 \mid 0,0,C)$, where $\lambda \in K$ and 
	$C \in \O_F$. First, we check that the action of our copy of $\Omega_{10}^{-,K}(F)$ 
	preserves $V_{10}^-$. The action on this subspace is given by
	\begin{equation*}
		\begin{array}{r@{\;}c@{\;}l}
			N_x: (\lambda\cdot 1_{\O}, -\lambda^{\sigma} \cdot 1_{\O}, 0 \mid 0,0, C) & \mapsto & 
				((\lambda - \Tr(C \ovx^{\s}))\cdot 1_{\O}, 
				-(\lambda^{\sigma} -\Tr(\ovC x))\cdot 1_{\O}, 
				0 \mid \\
		& &					0,0, C - x^{\s} \ovC x + \lambda x + \lambda^{\sigma} x^{\s}).
		\end{array}
	\end{equation*}
	Note that the product $x^{\s} \ovC x$ makes sense since $x$ and $x^{\s}$ generate a
	sociable subalgebra of $\O_K$. It is easy to see that 
	$(\lambda - \Tr(C\ovx^{\s}))^{\sigma} = 
	\lambda^{\sigma} - \Tr(\ovC x)$. Lemma \ref{lemma:sociable} also implies that 
	$x^{\s} \ovC x = x \ovC x^{\s} = (x^{\s} \ovC x)^{\s}$. That is, $x^{\s} \ovC x$ is an 
    element of $\O_F$. Finally, $\lambda x +  \lambda^{\sigma} x^{\s} \in \O_F$, 
    so we conclude that 
    the elements $N_x$ indeed preserve $V_{10}^{-}$.
    
	Note that the stabiliser preserves restrictions of both $\QQ$ and $\H$ on $V_{10}^-$. Proposition
	\ref{prop:3_2forms} asserts that such a subspace is unique, so we conclude that $G$ is 
	a subgroup of $\GO_{10}^{-,K}(F)$ in its action on $V_{10}^-$. In fact, since $G$ is a 
	subgroup of white vector stabiliser, namely $K^{16}\cn\Spin_{10}^+(K)$, we conclude that 
	$G \leqslant \SO_{10}^{-,K}(F)$ in its action on $V_{10}^-$. 
	
	Now let us look at the action on $V_{10}^{-}$ in more detail. The restriction of 
	$\inner{\cdot}{\cdot}_{\H}$ on
	this $10$-space is represented by the Gram matrix
	\begin{equation*}
		\begin{bmatrix}
			1 & \cdot & \cdot & \cdot & \cdot & \cdot & \cdot & \cdot & \cdot & \cdot \\
			\cdot & 1 & \cdot & \cdot & \cdot & \cdot & \cdot & \cdot & \cdot & \cdot \\
			\cdot & \cdot & \cdot & \cdot & \cdot & \cdot & \cdot & \cdot & \cdot & 1 \\
			\cdot & \cdot & \cdot & \cdot & \cdot & \cdot & \cdot & \cdot & 1 & \cdot \\
			\cdot & \cdot & \cdot & \cdot & \cdot & \cdot & \cdot & 1 & \cdot & \cdot \\
			\cdot & \cdot & \cdot & \cdot & \cdot & \cdot & 1 & \cdot & \cdot & \cdot \\
			\cdot & \cdot & \cdot & \cdot & \cdot & 1 & \cdot & \cdot & \cdot & \cdot \\
			\cdot & \cdot & \cdot & \cdot & 1 & \cdot & \cdot & \cdot & \cdot & \cdot \\
			\cdot & \cdot & \cdot & 1 & \cdot & \cdot & \cdot & \cdot & \cdot & \cdot \\
			\cdot & \cdot & 1 & \cdot & \cdot & \cdot & \cdot & \cdot & \cdot & \cdot \\
		\end{bmatrix}
	\end{equation*}
%	\begin{equation*}
%		\begin{bmatrix}
%			1 & 0 & 0 & 0 & 0 & 0 & 0 & 0 & 0 & 0 \\
%			0 & 1 & 0 & 0 & 0 & 0 & 0 & 0 & 0 & 0 \\
%			0 & 0 & 0 & 0 & 0 & 0 & 0 & 0 & 0 & 1 \\
%			0 & 0 & 0 & 0 & 0 & 0 & 0 & 0 & 1 & 0 \\
%			0 & 0 & 0 & 0 & 0 & 0 & 0 & 1 & 0 & 0 \\
%			0 & 0 & 0 & 0 & 0 & 0 & 1 & 0 & 0 & 0 \\
%			0 & 0 & 0 & 0 & 0 & 1 & 0 & 0 & 0 & 0 \\
%			0 & 0 & 0 & 0 & 1 & 0 & 0 & 0 & 0 & 0 \\
%			0 & 0 & 0 & 1 & 0 & 0 & 0 & 0 & 0 & 0 \\
%			0 & 0 & 1 & 0 & 0 & 0 & 0 & 0 & 0 & 0 \\
%		\end{bmatrix},
%	\end{equation*}
	which is block diagonal (here zeroes are replaced with dots). Consider the action  
	by an element $S$ such that it has the following matrix form when acting on $V_{10}^-$: 
	\begin{equation*}
		\newcommand*{\temp}{\multicolumn{1}{|r}{}}
		S_{10} = \left[
			\begin{array}{ccccc}
				\lambda & \cdot &\temp &  &  \\
				\cdot & \mu &\temp &  & \\ \hline
				 &  &\temp & & \\	
				 & &\temp & \ \II_8 & \\
				 & &\temp & &
			\end{array}
		\right],
	\end{equation*}
	where $\lambda \mu = 1$, i.e. $\mu = \lambda^{-1}$. Note that a generalised element 
	$P_{\lambda}$ acts on $\J$ in the following way:
	\begin{equation*}
		\begin{array}{r@{\;}c@{\;}l}
			P_{\lambda} : (a,b,c\mid A,B,C) & \mapsto & (\lambda^2 a , \lambda^{-2} b, c \mid
										\lambda^{-1}A, \lambda B, C).
		\end{array}
	\end{equation*}
	It follows that the action on $V_{10}^-$ of $S^2$ is the same as the action of $P_{\lambda}$, 
	so the action of $S$ on the $16$-space $\J_{16}^{cAB}$ is determined up to sign.
	
	Our element $S$ also commutes with $\Omega_{8}^{+}(F)$, generated by the actions of
	$P_u$, as $u$ ranges through all octonions of norm $1$ in $\O_F$. Therefore, the action
	of $S$ on $\J$ is given by 
	\begin{equation*}
		\newcommand{\temp}{\multicolumn{1}{r|}{}}
		S_{27} = \left[
			\begin{array}{cccccccccccccccc}
				\lambda & \cdot & \cdot & \temp & & & & & & & & & & & &  \\
				\cdot & \lambda^{-1} & \cdot & \temp & & & & & & & & & & & & \\
				\cdot & \cdot & 1 & \temp & & & & & & & & & & & & \\ \cline{1-8} 
				 & &  & \temp & & & &\temp & & & & & & & & \\
				 & &  & \temp & & \ \ \alpha \II_8 & &\temp & & & & & & & & \\
				 & &  & \temp & & & &\temp & & & & & & & & \\ \cline{5-12}
				 & &  &  & & & &\temp & & & & \temp  & & & & \\
				 & &  &  & & & &\temp & &\ \ \beta \II_{8}  & &\temp   & & & & \\
				 & &  &  & & & &\temp & &   & &\temp   & & & & \\	\cline{9-16}	
				 & &  &  & & & & & &   & &\temp   & & & & \\
				 & &  &  & & & & & &   & &\temp   & & \ \ \II_{8} &  & \\
				 & &  &  & & & & & &   & &\temp   & & & & \\
			\end{array}
		\right].
	\end{equation*}
	Since the action by $S^2$ coincides with the action by $P_{\lambda}$, we get
	\begin{equation*}
		\left. 
			\begin{array}{r@{\;}c@{\;}l}
				\alpha^2 & = & \lambda^{-1}, \\
				\beta^2 & = & \lambda. 
			\end{array}
		\right\}
	\end{equation*}
	Next, the action of $S$ preserves the Dickson--Freudenthal determinant:
	\begin{equation*}
		\left.
		\begin{array}{r@{\;}c@{\;}l}
			abc & \mapsto & \lambda \lambda^{-1} abc, \\
			a A\ovA & \mapsto & \lambda \alpha^2 a A\ovA, \\
			b B\ovB & \mapsto & \lambda^{-1} \beta^2 b B\ovB, \\
			c C\ovC & \mapsto & c C\ovC, \\
			\Tr(ABC) & \mapsto & \alpha \beta \Tr(ABC).
		\end{array}
		\right\}
	\end{equation*}
	Setting, for example $a=b=c=0$ and $A=B=1$, $C=e_0$, we obtain $\alpha \beta = 1$. 
	Similarly, by preserving the Hermitean form $\H$ 
	we obtain the following conditions on $\alpha, \beta$ and $\lambda$:
	\begin{equation*}
		\begin{array}{r@{\;}c@{\;}l}
			\lambda \lambda^{\sigma} & = & 1, \\
			\alpha \alpha^{\sigma} & = & 1, \\
			\beta \beta^{\sigma} & = & 1.
		\end{array}
	\end{equation*}
	We see that $S$ acts on $V_{10}^-$ as 
	\begin{equation}
		\newcommand{\temp}{\multicolumn{1}{|r}{}}
		S_{10} = \left[
			\begin{array}{ccccc}
				\lambda & \cdot &\temp &  &  \\
				\cdot & \lambda^{-1} &\temp &  & \\ \hline
				 &  &\temp & & \\	
				 & &\temp & \ \II_8 & \\
				 & &\temp & &
			\end{array}
		\right].
	\end{equation}
	This is an element of $\Omega_{10}^{-,K}(F)$. With the help of Lemma \ref{lemma:1_2_actions}
	we find that the action of the stabiliser in ${}^2\SE_6^K(F)$ of $X_3$ is that of 
	$\Spin_{10}^{-,K}(F)$. 
\end{proof}

\subsection{The stabiliser of type 1 vector}

We are ready to investigate the stabiliser in ${}^2\SE_6^K(F)$ of 
\mbox{$X_1 = (0,0,0\mid 0,0,e_0)$}. From the previous section we know that the stabiliser of 
$X_3$ is the subgroup of shape $\Spin_{10}^{-,K}(F)$. By stabilising $X_3$ and $X_1$ simultaneously,
with the help of Lemma \ref{lemma:A_stabiliser_omega}, we find that the stabiliser of $X_1$ is 
at least a subgroup $F^8\cn\Spin_8^{-,K}(F)$.

To find the rest of the stabiliser, we consider the elements $N_{\mu x}$, $N_{\mu x}'$, and
$N_{\mu x}''$ which preserve $X_1$ but do not preserve $X_3$. It turns out that only the elements
$N_{\mu x}'$ with $x \in \{e_{\bar{\omega}}, e_{\omega}, e_{-0}, e_1\}$ and 
$N_{\nu y}''$ with $y \in \{ e_{-1}, e_{-0}, e_{-\omega}, e_{-\bar{\omega}} \}$ for any 
$\mu,\nu \in K$,
do what we want, except when $(\mu,\nu) = (0,0)$. These elements move $X_3$ in the following way:
\begin{equation}
	\begin{array}{r@{\;}c@{\;}l}
		N_{\mu x}': (0,0,1\mid 0,0,0) & \mapsto & (0,0,1\mid -\mu^{\s} x, 0, 0), \\
		N_{\nu y}'': (0,0,1\mid 0,0,0) & \mapsto & (0,0,1\mid 0, \nu y, 0).  
	\end{array}
\end{equation}
Consider the subspace $U_1^{\perp}$ (with respect to $\inner{\cdot}{\cdot}_{\H}$). We find that
$U_1^{\perp}$ is $10$-dimensional and spanned by the vectors $(0,0,c\mid A,B,C)$ with
\begin{equation}
	\begin{array}{r@{\;}c@{\;}l}
		A & \in & \langle e_{\bar{\omega}}, e_{\omega}, e_{-0}, e_1 \rangle, \\
		B & \in & \langle e_{-1}, e_{-0}, e_{-\omega}, e_{-\bar{\omega}} \rangle, \\
		C & \in & \langle e_0 \rangle.
	\end{array}
\end{equation}
Note that $X_3 \in U_1^{\perp}$, so the stabiliser in ${}^2\SE_6^K(F)$ sends $X_3$ to some vector
in $U_1^{\perp}$. 
The actions on $U_1^{\perp}$ of the elements $N_{\mu x}'$ and $N_{\nu x}''$ from above are given
by
\begin{equation}
	\begin{array}{r@{\;}c@{\;}l}
		N_{\mu x}' : (0,0,c\mid A,B,C) & \mapsto & 
			(0,-\mu^{\s}\Tr(A\ovx^{\s}), c+\mu\Tr(\ovA x) \mid \\
		& & \mid A - \mu\mu^{\s} x^{\s} \ovA x - \mu^{\s} c x^{\s}, \\
		& &		 B + \mu\ovx\ovC, C - \mu^{\s} \ovB \ovx^{\s}), \\
				 
		N_{\nu y}'' : (0,0,c\mid A,B,C) & \mapsto & 
			(\nu\Tr(\ovB y), 0, c-\nu^{\s}\Tr(B\ovy^{s}) \mid \\
		& & \mid A - \nu^{\s}\ovC\ovy^{\s}, \\
		& & B - \nu\nu^{\s} y^{\s}\ovB y + \nu c y, C + \nu\ovy\ovA). 
	\end{array}
\end{equation}
Since $x \in \{e_{\bar{\omega}}, e_{\omega}, e_{-0}, e_1\}$, we find $\Tr(A\ovx^{\s}) = 
\Tr(\ovA x) = 0 = x^{\s} \ovA x = \ovx \ovC$, and also $x^{\s} = x$, $\ovB \ovx^{\s} \in 
\langle e_0 \rangle$. Similarly, $\Tr(\ovB y) = \Tr(B\ovy^{\s}) = 0 = y^{\s} \ovB y = \ovC \ovy^{\s}$,
and $y^{\s} = y$, $\ovy \ovA \in \langle e_0 \rangle$ as $y \in \{ e_{-1}, e_{-0}, e_{-\omega}, 
e_{-\bar{\omega}} \}$. After these simplifications the action takes the form
\begin{equation}
	\begin{array}{r@{\;}c@{\;}l}
		N_{\mu x}' : (0,0,c\mid A,B,C) & \mapsto &
			(0,0,c\mid A  - \mu^{\s} c x, B, C - \mu^{\s}\ovB\ovx ), \\
		
		N_{\nu y}'' : (0,0,c\mid A,B,C) & \mapsto &
			(0,0,c\mid A, B + \nu c y, C + \nu \ovy \ovA).
	\end{array}
\end{equation}
The value of $\H$ on $U_1^{\perp}$ is given by
\begin{equation}
	\H((0,0,c\mid A,B,C)) = c c^{\s},
\end{equation}
so we conclude that the subgroup of the stabiliser of $X_1$, which does not preserve $X_3$, 
sends $X_3$ to a vector of the form $(0,0,1\mid A,B, \ovB \ovA)$. These vectors span a 
$9$-subspace of $U_1^{\perp}$. We are interested in the action on its subspaces
$S_A$ and $S_B$, spanned by the vectors of the form $(0,0,1\mid A,0,0)$ and 
$(0,0,1\mid 0,B,0)$ respectively. The action of the elements $N_{\mu x}'$ and $N_{\nu y}''$ 
on these subspaces is given by
\begin{equation}
	\begin{array}{r@{\;}c@{\;}l}
		N_{\mu x}' : (0,0,1\mid A,0,0) & \mapsto & 
			(0,0,1\mid A -\mu^{\s} x , 0, 0), \\
			
		N_{\mu x}' : (0,0,1\mid 0,B,0) & \mapsto &
			(0,0,1\mid 0,B,0), \\
			
		N_{\nu y}'' : (0,0,1\mid A,0,0) & \mapsto &
			(0,0,1\mid A,0,0), \\
			
		N_{\nu y}'' : (0,0,1\mid 0,B,0) & \mapsto &
			(0,0,1\mid 0,B + \nu y, 0).
	\end{array}
\end{equation}
We see that the action of $N_{\mu x}'$ on $S_A$ is that of $K^4$, as well as the action
of $N_{\nu y}''$ on $S_B$. We also see that the intersection of these two copies of $K^4$
is trivial, hence we get the action of $K^4 \times K^4 \cong K^8$ on $U_1^{\perp}$. 
Finally, we have shown the following.

\begin{theorem}
	The stabiliser in ${}^2\SE_6^K(F)$ of $X_1 = (0,0,0\mid 0,0,e_0)$ is the subgroup
	of shape $F^8.K^8\cn\Spin_8^{-,K}(F)$. 
\end{theorem}

%As we know, $X_1$ determines the $17$-space $U_1$, spanned by the Albert vectors of the form
%$(a,b,0\mid A,B,C)$, where
%\begin{equation}
%	\begin{array}{r@{\;}c@{\;}l}
%		A & \in & \langle e_{\bar{\omega}}, e_{\omega}, e_{-0}, e_1 \rangle, \\
%		B & \in & \langle e_{-1}, e_{-0}, e_{-\omega}, e_{-\bar{\omega}} \rangle, \\
%		C & \in & \langle e_{-1}, e_{\bar{\omega}}, e_{\omega}, e_0,
%						e_{-\omega}, e_{-\bar{\omega}}, e_1 \rangle.
%	\end{array}
%\end{equation}
%With respect to $\H$, the subspace $U_1^{\perp}$ is spanned by the Albert vectors
%of the form \mbox{$(0,0,c \mid A, B, C)$} with
%\begin{equation}
%	\begin{array}{r@{\;}c@{\;}l}
%		A & \in & \langle e_{\bar{\omega}}, e_{\omega}, e_{-0}, e_1 \rangle, \\
%		B & \in & \langle e_{-1}, e_{-0}, e_{-\omega}, e_{-\bar{\omega}} \rangle, \\
%		C & \in & \langle e_0 \rangle.
%	\end{array}
%\end{equation}
%It follows that $U_1^{\perp}$ is a $10$-space of $\J$. Given $W \in U_1^{\perp}$, the value 
%of $\H$ is given by
%\begin{equation}
%	\H(W) = c c^{\s} + \Tr(A \ovA^{\s}) + \Tr(B \ovB^{\s} ).
%\end{equation}
%Notice that $\Tr(A \ovA^{\s}) = \Tr(B \ovB^{\s}) = 0$, so in fact
%\begin{equation}
%	\H(W) = c c^{\s}.
%\end{equation}
%We also see that $X_3 \in U_1^{\perp}$. An Albert vector $(0,0,c\mid A,B,C) \in U_1^{\perp}$ 
%is white if and only if 
%\begin{equation}
%	\left\{
%		\begin{array}{l}
%			A\ovA = B\ovB = C\ovC = 0, \\
%			BC = CA = 0, \\
%			AB = c \ovC. 
%		\end{array}
%	\right.
%\end{equation}
%The first two are satisfied automatically, so we get $AB = c \ovC$. If $c \neq 0$, it is possible
%to find a unique value of $C$ such that $(0,0,c\mid A,B,C)$ is white and non-isotropic with respect
%to $\H$. 

\subsection{The stabiliser of type 2 vector}

Finally, we investigate the stabiliser in ${}^2\SE_6^K(F)$ of 
\mbox{$X_2 = (0,0,0 \mid 0,0,e_0 + \lambda e_1)$}, where $\lambda \in K \setminus F$. 
For this we need to prepare certain ingredients. 
Recall that the $17$-space $U_1$ determined by $X_1$ is spanned by the vectors of the form
$(a,b,0\mid A,B,C)$ with 
\begin{equation}
	\begin{array}{r@{\;}c@{\;}l}
		A & \in & \langle e_{\bar{\omega}}, e_{\omega}, e_{-0}, e_{1} \rangle, \\
		B & \in & \langle e_{-1}, e_{-0}, e_{-\omega} e_{-\bar{\omega}} \rangle, \\
		C & \in & \langle e_{-1}, e_{\bar{\omega}}, e_{\omega} e_0, 
					e_{-\omega}, e_{-\bar{\omega}}, e_{1} \rangle. 
	\end{array}
\end{equation}
Recall that the stabiliser of a white vector in $\SE_6(K)$ is a subgroup of shape 
$K^{16}\cn\Spin_{10}^+(K)$ (Theorem \ref{theorem:1_white_stab}). As we know, an element
in the stabiliser, preserving the $1\oplus 16 \oplus 10$ decomposition, belongs to 
$\Spin_{10}^+(K)$. The $10$-space of $X_1$, denoted $W_1$, 
preserved by such an element, is spanned by the
Albert vectors $(0,0,c \mid A,B,C)$ with
\begin{equation}
	\begin{array}{r@{\;}c@{\;}l}
		A & \in & \langle e_{-1}, e_0, e_{-\omega}, e_{-\bar{\omega}} \rangle, \\
		B & \in & \langle e_{\bar{\omega}}, e_{\omega}, e_0, e_1 \rangle, \\
		C & \in & \langle e_{-0} \rangle.
	\end{array}
\end{equation}

To obtain the corresponding $10$-space for $X_2$, we 
find an element in $\SE_6(K)$ which maps $X_1$ to $X_2$. For example, the element 
$P_u$ with $u = 1 - \lambda e_1$ does exactly what we want. Now, $W_2$ is spanned
by the vectors of the form $(0,0,c \mid A,B,C)$ with
\begin{equation}
	\begin{array}{r@{\;}c@{\;}l}
		A & \in & \langle e_{-1}+\lambda e_{-0}, e_0 - \lambda e_1, e_{-\omega}, e_{-\bar{\omega}}
					\rangle, \\
					
		B & \in & \langle e_{\bar{\omega}}-\lambda e_{-\omega}, e_{\omega} + \lambda e_{-\bar{\omega}},
					e_0, e_1 \rangle, \\
					
		C & \in & \langle e_{-0} + \lambda e_1 \rangle.
	\end{array}
\end{equation}
Note that since $P_{1-\lambda e_1}$ is an element of $\SE_6(K)$, it preserves colour, so all the 
white vectors in $W_1$ are mapped to the white vectors in $W_2$.
An Albert vector \mbox{$(0,0,c \mid A,B,C) \in W_1$} is white if and only if
 \mbox{$A\ovA = B\ovB = C\ovC = 0$, $AB = c\ovC$}, and $BC = CA = 0$. Most of these conditions,
except possibly $AB = c\ovC$, are satisfied automatically. Now, we find that $AB \in \langle e_0
\rangle$ by an explicit calculation, so if $c \neq 0$, there is a unique value of $C$ such that 
an element $(0,0,c \mid A,B,C)$ in $W_2$ is white. The action of $P_{1-\lambda e_1}$ on the third 
`co\"{o}rdinate' is trivial, so we may conclude that there is a unique value of $C$ such that 
an element $(0,0,c \mid A,B,C)$ in $W_2$ is white given $c \neq 0$. 

Next, the subgroup of $\Spin_{10}^+(K)$, stabilising $X_2$, not only preserves the colour, 
but it also preserves the value 
of $\H$ on $W_2$. For $X = (0,0,c\mid A,B,C) \in W_2$, 
the value of $\H$ is given by
\begin{equation}
	\H(X) = c c^{\s} + \Tr(A\ovA^{\s}) + \Tr(B\ovB^{\s}) + \Tr(C\ovC^{\s}). 
\end{equation}
As we know,
\begin{equation}
	\begin{array}{r@{\;}c@{\;}l}
		A & = & A_{-1} (e_{-1} + \lambda e_{-0}) + A_0 (e_0 - \lambda e_1) + 
				A_{-\omega} e_{-\omega} + A_{-\bar{\omega}} e_{-\bar{\omega}}, \\
				
		B & = & B_{\bar{\omega}} ( e_{\bar{\omega}} - \lambda e_{-\omega} ) +
				B_{\omega} ( e_{\omega} + \lambda e_{-\bar{\omega}} ) + 
				B_0 e_0 + B_1 e_1, \\
				
		C & = & C_{-0} ( e_{-0} + \lambda e_1 ).
	\end{array}
\end{equation}
We then find
\begin{multline}
	A\ovA^{\s} = A_{-1} A_0^{\s} ( e_{-1} + \lambda e_{-0} - \lambda^{\s} e_0 + 
						\lambda \lambda^{\s} e_1 ) - 
	A_{-1} A_{-\omega}^{\s} e_{\bar{\omega}} + A_{-1} A_{-\bar{\omega}}^{\s} e_{\omega} \\	
	+ A_0 A_{-1}^{\s} (\lambda^{\s} e_0 - \lambda \lambda^{\s} e_1 - e_{-1} - \lambda e_{-0} ) -
	A_0 A_{-\omega}^{\s} e_{-\omega} - A_0 A_{-\bar{\omega}}^{\s} e_{-\bar{\omega}} \\	
	+ A_{-\omega} A_{-1}^{\s} e_{\bar{\omega}} + A_{-\omega} A_0^{\s} e_{-\omega}  - 
	A_{-\omega} A_{-\bar{\omega}}^{\s} e_1 \\
	+ A_{-\bar{\omega}} A_{-\omega}^{\s} e_{-\bar{\omega}} + 
	A_{-\bar{\omega}} A_{-1}^{\s} e_{\omega} + A_{-\bar{\omega}} A_{-\omega}^{\s} e_1,
\end{multline}
so 
\begin{multline}
	\Tr(A\ovA^{\s}) = -\lambda^{\s} A_{-1} A_0^{\s} + \lambda^{\s} A_0 A_{-1}^{\s} + 
						\lambda A_{-1} A_0^{\s} - \lambda A_0 A_{-1}^{\s} \\
						= (\lambda^{\s} - \lambda)( A_0 A_{-1}^{\s} - A_{-1} A_0^{\s} ).
\end{multline}
Next,
\begin{multline}
	B\ovB^{\s} = -B_{\bar{\omega}}B_{\omega}^{\s} ( -e_{-1} + \lambda e_0 - \lambda^{\s} e_{-0}
					- \lambda \lambda^{\s} e_1) - B_{\bar{\omega}} B_0^{\s} \lambda e_{-\omega} 
					+ B_{\bar{\omega}} B_1^{\s} e_{-\omega} \\
	+ B_{\omega}B_{\bar{\omega}}^{\s} ( -\lambda^{\s} e_{-0} -\lambda \lambda^{\s} e_1 - e_{-1}
					+ \lambda e_0) + B_{\omega}B_0^{\s} \lambda e_{-\bar{\omega}} - 
					B_{\omega} B_1^{\s} e_{-\bar{\omega}} \\
	+ B_0 B_{\bar{\omega}}^{\s} \lambda^{\s} e_{-\omega} - B_0 B_{\omega}^{\s} \lambda^{\s}
					e_{-\bar{\omega}} + B_1 B_{\bar{\omega}}^{\s} e_{-\omega}
					- B_1 B_{\omega}^{\s} e_{-\bar{\omega}},
\end{multline}
and so similarly,
\begin{equation}
	\Tr(B\ovB^{\s}) = (\lambda^{\s} - \lambda) ( B_{\bar{\omega}} B_{\omega}^{\s} - 
					B_{\omega} B_{\bar{\omega}}^{\s} ).
\end{equation}
Finally,
\begin{equation}
	C\ovC^{\s} = C_{-0}C_{-0}^{\s}(\lambda e_1 - \lambda^{\s}e_1),
\end{equation}
so
\begin{equation}
	\Tr(C\ovC^{\s}) = 0.
\end{equation}
It follows that on the elements of $W_2$, Hermitian form $\H$ becomes
\begin{equation}
	\H(X) = c c^{\s} + (\lambda^{\s} - \lambda) ( A_0 A_{-1}^{\s} - A_{-1} A_0^{\s} + 
			B_{\bar{\omega}} B_{\omega}^{\s} - B_{\omega} B_{\bar{\omega}}^{\s}).
\end{equation}
This is a unitary form in $5$ variables, which is preserved by the subgroup of the stabiliser of $X_2$
inside $\Spin_{10}^+(K)$. Note that
$\lambda^{\s}-\lambda \neq 0$ since $\lambda \in K \setminus F$. Given $X = (a,b,c \mid A,B,C) \in \J$,
the value of the quadratic form $Q(X) = \fdet(X+X_2) - \fdet(X)$, determined by $X_2$, is 
\begin{multline}
	Q(X) = \Tr( (e_0 + \lambda e_1) ( AB - c\ovC ) ) \\
	= A_0 (B_0 - \lambda B_{-1}) - (A_{-\omega} - \lambda A_{\bar{\omega}}) B_{\omega} \\ 
	- (A_{-\bar{\omega}} + \lambda A_{\omega}) B_{\bar{\omega}} - A_{-1} (B_1 + \lambda B_{-0}) \\
	- c (C_{-0} + \lambda C_{-1}).
\end{multline}
On $W_2$ this becomes
\begin{equation}
	Q(X) = A_0 B_0 - A_{-\omega}B_{\omega} - A_{-\bar{\omega}}B_{\bar{\omega}} - A_{-1}B_1 - cC_{-0}. 
\end{equation}
Consider the $5$-dimensional subspace of $W_2$ spanned by the vectors $(0,0,c\mid A,B,C)$ with
\begin{equation}
	\begin{array}{r@{\;}c@{\;}l}
		A & \in & \langle e_{-\omega}, e_{-\bar{\omega}} \rangle, \\
		B & \in & \langle e_0, e_1 \rangle, \\
		C & \in & \langle e_{-0} \rangle. 
	\end{array}
\end{equation}
This is the radical of $\H$ in $W_2$. We also notice that this subspace coincides with a 
maximal totally isotropic subspace of $Q$ on $W_2$. Therefore, the action of the subgroup
of $\Spin_{10}^+(K)$, preserving $W_2$, also preserves a $5 \oplus 5$ decomposition of $W_2$.
As a result, we get the action of the group of type $\SU_5^K(F,\H)$.

Using the element $P_{1 - \lambda e_1}$ we can also easily find the $17$-space $U_2$, determined
by $X_2$. We conjugate $U_1$ by $P_{1-\lambda e_1}$, and find that $U_2$ is spanned by the vectors
of the form $(a,b,0\mid A,B,C)$ with
\begin{equation}
	\begin{array}{r@{\;}c@{\;}l}
		A & \in & \langle e_{\bar{\omega}}+\lambda e_{-\omega}, e_{\omega}-\lambda e_{-\bar{\omega}}
		, e_{-0}, e_1
		\rangle, \\
		B & \in & \langle e_{-1}+\lambda e_0, e_{-0} - \lambda e_1,
	e_{-\omega}, e_{-\bar{\omega}}  \rangle, \\
		C & \in & \langle e_{-1}-\lambda^2 e_1 - \lambda \cdot 1_{\O}, e_{\bar{\omega}}, e_{\omega},
	e_0 + \lambda e_1, e_{-\omega}, e_{-\bar{\omega}}, e_1 \rangle.
	\end{array}
\end{equation}
The subspace $U_2$ is preserved by the stabiliser of $W_2$, and therefore so is the orthogonal 
complement $U_2^{\perp}$ in $\J$, 
which is $10$-dimensional and spanned by the Albert vectors of the form
$(0,0,c \mid A,B,C)$ with
\begin{equation}
	\begin{array}{r@{\;}c@{\;}l}
		A & \in & \langle e_{\bar{\omega}} + \lambda^{\s} e_{-\omega}, e_{\omega} - \lambda^{\s} 
					e_{-\bar{\omega}}, e_{-0}, e_1 \rangle, \\
		
		B & \in & \langle e_{-1} + \lambda^{\s} e_0, e_{-0} - \lambda^{\s} e_1, 
					e_{-\omega}, e_{-\bar{\omega}} \rangle, \\
					
		C & \in & \langle e_0 + \lambda^{\s} e_1 \rangle. 
	\end{array}
\end{equation}
As a result, we get the action of $K^{10}\cn\SU_5^K(F,\H)$. The $\SU_5^K(F,\H)$ acts 
on the corresponding image of $W_2$, which is being moved around. 

Now assume the trivial action on $U_2^{\perp}$. Then the action is also trivial on 
$U_2^{\perp} \cap U_2$, and so it is trivial on $(U_2^{\perp} \cap U_2)^{\perp}$,
and finally, the same holds for $W_2 \cap (U_2^{\perp} \cap U_2)^{\perp}$. 
The latter is a $5$-dimensional subspace of $W_2$, spanned by the vectors
$(0,0,c\mid A,B,0)$ with
\begin{equation}
	\begin{array}{r@{\;}c@{\;}l}
		A & \in & \langle e_{-\omega}, e_{-\bar{\omega}} \rangle, \\
		B & \in & \langle e_{0}, e_1 \rangle.
	\end{array}
\end{equation}

 

\section{The case of a finite field}
\subsection{White vectors in $\J_8^C$}

As a practical counting excersise, we count the isotropic white vectors in $\J_{8}^{C}$.
As before, $K = \Fqs$. First, we need the following auxiliary result.

\begin{lemma}
		\label{lemma:zn}
		Let $V$ be a vector space over $\Fqs$ of dimension $2m$. 
		Define the map $Z_m: V \rightarrow \Fq$ in the following way:
		\begin{equation*}
			Z_m(x) = (x_1^q - x_1) (x_2^q - x_2) + (x_3^q - x_3) (x_4^q - x_4) + \cdots +
			(x_{2m-1}^q-x_{2m-1}) (x_{2m}^q - x_{2m}),
		\end{equation*}
		where $x = (x_1, ..., x_{2m})$. 
		Denote by $z_m$ the number of $x\in V$ such that $Z_m(x) = 0$. Then
		\begin{equation*}
			z_n = q^{3m-1}(q^m+q-1).
		\end{equation*}
\end{lemma}

\begin{proof}
		We proceed by induction on $m$. If $m=1$, the equality $Z_m(x)=0$ reduces to
		\begin{equation*}
			(x_1^q - x_1) (x_2^q - x_2) = 0.
		\end{equation*}
		Note that this is possible when $x_1^q = x_1$ or $x_2^q = x_2$, i.e. when
		$x_1 \in \Fq$ or $x_2 \in \Fq$. Thus, when $m=1$ there are precisely
		$2q^3-q^2 = q^2(q+q-1)$ solutions.

		Now suppose that the statement holds for all integers $k$ such that
		$1 \leqslant k \leqslant m-1$. In the case
		\begin{equation*}
			\left.
			\begin{array}{l}
				(x_1^q - x_1) (x_2^q - x_2) = 0, \\
				(x_3^q - x_3) (x_4^q - x_4) + \cdots + 
					(x_{2m-1}^q-x_{2m-1}) (x_{2m}^q - x_{2m}) = 0,
			\end{array}
			\right\}
		\end{equation*}
		we get $z_1 z_{m-1}$ solutions. On the other hand, if
		\begin{equation*}
			\left.
			\begin{array}{l}
				(x_1^q - x_1) (x_2^q - x_2) = \lambda, \\
				(x_3^q - x_3) (x_4^q - x_4) + \cdots + (x_{2m-1}^q-x_{2m-1})
						(x_{2m}^q - x_{2m}) = -\lambda
			\end{array}
			\right\}
		\end{equation*}
		for $0\neq \lambda \in \Fq$, there are 
		\begin{equation*}
			(q^4-z_1) \frac{(q^{4(m-1)}-z_{m-1})}{q-1}
		\end{equation*}
		solutions.
		We divide the second factor by $(q-1)$ since each pair $(x_1,x_2)$ satisfying the
		first equation, fixes the value of $\lambda$ for the second equation. Overall we have
		\begin{equation*}
			z_m = z_1 z_{m-1} + (q^4-z_1) \frac{(q^{4(m-1)}-z_{m-1})}{q-1}.
		\end{equation*}
		Thus, we have obtained a recurrence relation and by substituting $z_1$ and $z_{m-1}$ in
		it, we finally obtain $z_m = q^{3m-1}(q^m+q-1)$.
\end{proof}

The following theorem allows us to count the elements of $V$ satisfying simultaneously a 
certain quadratic and a certain Hermitean form.

	\begin{theorem}
		Let $V$ be an vector space over $\Fqs$ of dimension $2m$. 
		Let the quadratic form $Q_m:V \rightarrow \Fqs$ be defined as
		\begin{equation}
			Q_m(x) = x_1 x_2 + x_3 x_4 + \cdots + x_{2m-1} x_{2m},
		\end{equation}
		where $x = (x_1, ..., x_{2m})$, 
		and also define the Hermitean form $H_m:V \rightarrow \Fq$ by
		\begin{equation}
			H_m(x) = x_1^q x_2 + x_1 x_2^q + x_3^q x_4 + x_3 x_4^q + \cdots + x_{2m-1}^q x_{2m}
				+ x_{2m-1} x_{2m}^q.
		\end{equation}
		Let $w_m$ be the number of $x \in V$ such that
		\begin{equation}
			\label{eq:system1}
			\left.
				\begin{array}{l}
					Q_m(x) = 0, \\
					H_m(x) = 0.
				\end{array}
			\right\}
		\end{equation}
		Then
		\begin{equation}
			w_m = q^{2m} + q^{2m-1}(q^m-1)(q^{m-2}+1).
		\end{equation}
	\end{theorem}

	\begin{proof}
		We again proceed by induction on $m$. When $m=1$, the system (\ref{eq:system1}) reduces to
		\begin{equation*}
			\left.
				\begin{array}{l}
					x_1 x_2 = 0, \\
					x_1^q x_2 + x_1 x_2^q = 0.
				\end{array}
			\right\}
		\end{equation*}
		Note that each pair $(x_1,x_2)$ which satisfies the first equation also satisfies the second
		one, so in this case the number of solutions is $2q^2-1 = q^2 + q(q-1)(q^{-1}+1)$.

		Suppose now that the statement holds for all integers $k$ such that
		$1 \leqslant k \leqslant m-1$ and consider the following system:
		\begin{equation*}
			\left.
				\begin{array}{l}
					x_1 x_2 + x_3 x_4 + \cdots + x_{2m-1} x_{2m} = 0, \\
					x_1^q x_2 + x_1 x_2^q + x_3^q x_4 + x_3 x_4^q + \cdots + x_{2m-1}^q x_{2m}
				+ x_{2m-1} x_{2m}^q = 0.
				\end{array}
			\right\}
		\end{equation*}
		We distinguish two cases.

		First, consider the case $x_1 = 0$. Then $x_2$ can take any of the $q^2$ possible values
		and the remaining system is equivalent to
		\begin{equation*}
			\left.
				\begin{array}{l}
					Q_{m-1}(x) = 0, \\
					H_{m-1}(x) = 0,
				\end{array}
			\right\}
		\end{equation*}
		so there are $q^2 w_{m-1}$ solutions in this case.

		Now suppose that $x_1 \neq 0$. Without loss of generality we may consider the case $x_1=1$.
		The system (\ref{eq:system1}) takes the form
		\begin{equation*}
			\left.
				\begin{array}{l}
					x_2 = - x_3 x_4 - \cdots - x_{2m-1} x_{2m}, \\
					x_2 + x_2^q + x_3^q x_4 + x_3 x_4^q + \cdots + x_{2m-1}^q x_{2m}
				+ x_{2m-1} x_{2m}^q = 0.
				\end{array}
			\right\}
		\end{equation*}
		We substitute $x_2$ from the first equation into the second one to obtain
		\begin{equation*}
			(x_3^q-x_3)(x_4^q-x_4) + \cdots + (x_{2m-1}^q-x_{2m-1})(x_{2m}^q-x_{2m}) = 0.
		\end{equation*}
		Using the result of Lemma \ref{lemma:zn}, we obtain that in this case there are
		$(q^2-1)z_{m-1}$ solutions. In total, we obtain the following recurrence relation:
		\begin{equation*}
			w_n = q^2 w_{m-1} + (q^2-1) z_{m-1}.
		\end{equation*}
		By substituting the appropriate values for $w_{m-1}$ and $z_{m-1}$, we obtain the result.
	\end{proof}

An Albert vector $(0,0,0\mid 0,0,C) \in \J_8^C$ is white if and only if $C\ovC = 0$. Recall that
$\O_K$ is split, so we can use our favourite basis $\{\ e_i\ \big|\ i \in \pm I\ \}$. Note that
with respect to this basis $C\ovC = 0$ is equivalent to 
\begin{equation}
	C_{-1}C_{1} + C_{\bar{\omega}} C_{-\bar{\omega}} + C_{\omega} C_{-\omega} + C_{-0} C_{0} = 0.
\end{equation}
Next, $(0,0,0\mid 0,0,C)$ is isotropic if and only if $\Tr(C \ovC^{\s}) = 0$, which is equivalent
to
\begin{equation}
	C_{-1}^qC_{1} + C_{-1}C_{1}^q + 
	C_{\bar{\omega}}^q C_{-\bar{\omega}} + C_{\bar{\omega}} C_{-\bar{\omega}}^q + 
	C_{\omega}^q C_{-\omega} + C_{\omega} C_{-\omega}^q + 
	C_{-0}^q C_{0} + C_{-0} C_{0}^q = 0.
\end{equation}

We know that there are exactly $(q^8-1)(q^6+1)$ white vectors in $\J_8^C$. Furthermore,
there are
\begin{equation}
	w_4 - 1 = (q^2+1)(q^3+1)(q^3(q^2+1)(q-1)+(q^5+1))
\end{equation}
isotropic white vectors of the form $(0,0,0 \mid 0,0,C)$ and
\begin{equation}
q^6(q^4-1)(q^3-1)(q-1)
\end{equation}
non-isotropic white vectors of the same form.

Now, using Proposition \ref{prop:3_2forms_orbits}, we find that the full subgroup of 
$\SE_6(K)$ which preserves $\QQ$ 
and
$\H$ on $\J_8^C$ has three orbits on white points, and the sizes of these orbits are given by
\begin{enumerate}[(i)]
	\item $(q^4-1)(q^3+1)/(q-1)$;
	\item $q(q^6-1)(q^4-1)(q^2+1)/(q^2-1)$;
	\item $q^6 (q^4-1)(q^3-1)/(q+1)$.
\end{enumerate}
Of these, the first two are isotropic, while the last one is non-isotropic. 

\subsection{White vectors in $\J_{16}^{BC}$}

We can also count the isotropic white vectors in $\J_{16}^{BC}$, $K = \Fqs$. Suppose 
\mbox{$X = (0,0,0\mid 0,B,C)$} is white and note that the whiteness conditions 
take form $B\ovB = C\ovC = 0 = BC$. 

First, we count the white vectors $(0,0,0\mid 0,B,C)$ such that $B \neq 0$ and $C \neq 0$.
We notice that given $B \neq 0$ and $B\ovB = 0 = BC$, we automatically have $C\ovC = 0$. 
Indeed, for if $C\ovC \neq 0$, $C$ is invertible and $BC = 0$ implies $B = 0$, a contradiction.
So, there are $(q^8-1)(q^6+1)$ choices for $B$ and $(q^8-1)$ choices for $C$ 
(see Lemma \ref{prop:octonion_annihilator}). In total, there are $(q^8-1)^2 (q^6+1)$ white
vectors with $B \neq 0$ and $C \neq 0$. 

To count the isotropic white vectors of the form $(0,0,0\mid 0,B,C)$ we distiguish two cases:
\begin{equation}
	\label{eq:8twocases}
	\left.
		\begin{array}{l}
			B\ovB = BC = 0,\\
			\Tr(B\ovB^{\s}) = 0,\\
			\Tr(C\ovC^{\s}) = 0,\\
		\end{array}
	\right\},\quad
	\left.
		\begin{array}{l}
			B\ovB = BC = 0,\\
			\Tr(B\ovB^{\s}) = \lambda \neq 0,\\
			\Tr(C\ovC^{\s}) = -\lambda.\\
		\end{array}
	\right\}.
\end{equation}

In the previous section we learned that the subgroup of $\SE_6(K)$, preserving $\QQ$ and $\H$
on $\J_8^C$ has two orbits on isotropic white points. Proposition \ref{prop:3_2forms_orbits}
the first orbit consists of white points
$\langle X \rangle$ where $X$ is written over $\Fq$. That is, its representatives are
one-dimensional $\Fqs$-subspaces generated by the white vectors written over $\Fq$.
Suppose $X_C = (0,0,0\mid 0,0,C)$ belongs to the first orbit. By taking a particular candidate 
for $C$, it is easy to see that in this case there are $q^8-1$ choices for $B$.
Now, if $X_C$ belongs to the second orbit, there are $q^4(q^3+q^2-q)-1$ choices for $B$.
If, on the other hand, $X_C$ is non-isotropic, then there are $q^3(q^4-1)$ choices for $B$, and overall
we have 
\begin{multline}
	(q^4-1)(q^3+1)(q+1)(q^8-1) \\
	+\ q(q^6-1)(q^4-1)(q^2+1)(q^4(q^3+q^2-q)-1) \\
	+\ q^6(q^4-1)(q^3-1)(q-1)q^3(q^4-1) \\
	=\ (q^{13}+q^{11}-q^{10}+2q^9+q^8+2q^7-q^6+2q^4+1)(q^4-1)^2
\end{multline}
isotropic white vectors of the form $(0,0,0\mid 0,B,C)$ with $B,C \neq 0$. 

Recall that the group preserving $Q_4$ and $H_4$ has two orbits on the isotropic white
points in $J_8$ with
$(q^4-1)(q^3+1)/(q-1)$ and $q(q^6-1)(q^4-1)(q^2+1)/(q^2-1)$ elements. Again, by PropositionThe first orbit
consists of white points
$\langle X \rangle$ where $X$ is written over $\Fq$. That is, its representatives are
one-dimensional $\Fqs$-subspaces generated by the white vectors written over $\Fq$.
Since we know the totality of white vectors of this form, we find that there are
\begin{equation}
	q^6(q^4-1)^2(q^4+2)(q^3-1)(q-1)
\end{equation}
non-isotropic white vectors in $\J_{16}^{BC}$ with $B,C\neq 0$. Overall there are
\begin{multline}
	q^6(q^4-1)^2(q^4+2)(q^3-1)(q-1)\\
	+\ 2q^6(q^4-1)(q^3-1)(q-1)\\
	=\ q^{10}(q^8-1)(q^3-1)(q-1)
\end{multline}
non-isotropic white vectors in $\J_{16}^{BC}$.