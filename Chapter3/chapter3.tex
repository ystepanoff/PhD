\chapter{Groups of type ${}^2\EE_6$}
\ifpdf
    \graphicspath{{Chapter3/Chapter3Figs/PNG/}{Chapter3/Chapter3Figs/PDF/}{Chapter3/Chapter3Figs/}}
\else
    \graphicspath{{Chapter3/Chapter3Figs/EPS/}{Chapter3/Chapter3Figs/}}
\fi
%leave these commands here

\section{Quadratic field extensions}

Let $F$ and $K$ be two fields such that $F$ is a subfield of $K$. We say that 
$K$ is an \textit{extension field} of $F$. The \textit{degree} of $K$ over $F$, 
denoted by $[K:F]$, is the dimension of $K$ as a vector space over $F$. We denote the
extension of $F$ by $K$ as $K / F$.

A non-zero polynomial $f \in F[x]$ is called \textit{separable}, if each root of $f$ has 
multiplicity $1$. If $\alpha \in F$ is algebraic, that is, $\alpha$ is a root of some polynomial
$g \in F[x]$, then $\alpha$ is called \textit{separable}, if its minimal polynomial is separable. 

We have defined what it means for a field element and a polynomial to be separable. Suppose that
$[K:F]$ is finite, then $K / F$ is a \textit{separable} extension, if every element of $K$ is 
separable over $F$. We also say that $K / F$ is \textit{normal}, if every irreducible 
polynomial $f \in F[x]$ that has a root in K, splits into linear factors in $K[x]$. An extension
$K / F$ which is normal, separable, and of finite degree, is called a \textit{Galois extension}.
We are not going into too much detail here, for there are various  well-known references on theory
of field extensions, of very high quality, for instance, \cite{PeterCameron}, \cite{DummitFoote}, \cite{Lang}, or \cite{Stewart}. In this chapter
we are interested in Galois extensions of degree $2$. 
In case when $F = \Fq$, we have $|K| = q^2$, and so $K = \Fqs$ (see, for example, \cite{Moore}).

Given an extension $K / F$, we define the \textit{Galois group} of $K$ over $F$, denoted
$\Gal(K / F)$, to be the group
of automorphisms of $K$ that fix $F$ elementwise. In other words,
\begin{equation}
	\Gal(K / F) = \{\ \varsigma \in \Aut(K)\ \big|\ \alpha^{s} = \alpha\ \mbox{for all}\ 
		\alpha \in F\ \}.
\end{equation}
The most important result for us here is the following theorem.
\begin{theorem}
	Let $K / F$ be a Galois extension. Then
	\begin{equation}
		| \Gal(K / F) | = [K : F].
	\end{equation}
\end{theorem}
It follows that in case $[K:F] = 2$ there is a unique non-trivial field automorphism 
$\varsigma : K \rightarrow K$, fixing $F$ elementwise. If $F = \Fq$ and $K = \Fqs$, then
$s$ takes the form $s : \lambda \mapsto \lambda^q$. 


% Moore: Doubly-infinite systems of simple groups

\section{Spaces with two forms}

Let $K / F$ be a Galois extension of degree $2$, 
and let $s$ be a $K$-automorphism
with $F$ being its fixed field. Let also $V$ be a $2m$-dimensional vector space over $K$ with 
a quadratic form $Q$ and a conjugate-symmetric sesquilinear form 
\mbox{$B : V\times V \rightarrow K$},
defined with respect to $\varsigma$. Denote by $f$ the polar form of $Q$, and suppose that 
\mbox{$\mathcal{B} = \{ e_1, f_1, e_2, f_2, ..., e_m, f_m \}$}
is a hyperbolic basis of $V$ with respect to $f$, i.e.
\begin{equation}
	Q(e_i) = Q(f_i) = 0,\ f(e_i, f_i) = 1,
\end{equation}
and $f(e_i,e_j) = f(e_i,f_j) = 0$ for $i \neq j$. The above said means that $(V,Q)$ is a hyperbolic
orthogonal space. Denote by $G$ the maximal amongst all the subgroups of 
$\GO(V, Q)$ which preserve $B$. If $U$ 
is a subspace of $V$, then we denote the restrictions of $f$ and $B$ on $U$ as $f_U$ and $B_U$ 
respectively. 

\begin{definition}
	A $(Q,B)$-subspace of $V$ is an $F$-subspace $U$ of $V$ such that \mbox{$V = U \otimes_F K$}, 
	$f_U = B_U$ is an $F$-form on $U$, and $Q_U$ is a non-degenerate quadratic form on $U$ of 
	Witt index at least $2$. 
\end{definition}

\begin{proposition}
	If $U$ is a $(Q,B)$-subspace of $V$, then it is the unique $(Q,B)$-subspace of $V$, and
	$G = \GO(U,Q)$. 
\end{proposition}

We say that an element $v \in V$ is \textit{singular isotropic}, if $Q(v) = B(v,v) = 0$. 

\begin{proposition}
	\label{prop:3_2forms_orbits}
	Let $U$ be a $(Q,B)$-subspace of $V$ if Witt index at least $2$. Fix $\lambda \in K 
	\setminus F$. The group $G$ has two orbits on doubly singular points with representatives
	$\langle u \rangle$ and $\langle u + \lambda v \rangle$, where $u,v \in U$ and 
	$\langle u, v \rangle$ 	is a singular line. 
\end{proposition}

\section{Hermitean form in $\J$ and the group ${}^2\SE_6(K/F)$}

Suppose as before $K / F$ is a quadratic Galois extension with $F$ being a fixed field of a
$K$-automorphism $\varsigma$. In this chapter $\O_F$ will always be a split octonion algebra 
over $F$ and $\O_K = \O_F \otimes_F K$. We also use the same basis 
$\{\ e_i\ \mid\ i \in \pm I\ \}$ as in Section \ref{section:split_basis}.

Denote by $\s$ the automorphism of $\O_K$ induced by the field automorphism 
$\varsigma$:
\begin{equation}
	\left( \sum_{i\in \pm I} \lambda_i e_i \right)^{\s} = 
	\sum_{i\in \pm I} \lambda_{i}^{\varsigma} e_i. 
\end{equation}
Consider the following Hermitean form defined on the elements of $\O_K$:
\begin{equation}
	h(x) = A \ovA^{\s} + A^{\s} \ovA = \Tr(A \ovA^{\s}).
\end{equation}
On the Albert space $\J = \J_K$ this induces the Hermitean form $\H$, where
\begin{equation}
	\H( (a,b,c\mid A,B,C) ) = 
		a a^{\s} + b b^{\s} + c c^{\s} + 
		\Tr(A \ovA^{\s} + B\ovB^{\s} + C\ovC^{\s}).
\end{equation}

Using the construction from previous chapter, we obtain the group $\SE_6(K)$ in the usual 
way. Now define the group ${}^2\SE_6(K/F)$ as the subgroup of $\SE_6(K)$ which preserves 
$\H$. In case $F = \Fq$ and $K = \Fqs$, we denote this by ${}^2\SE_6(q)$. As before,
the group ${}^2\E_6(F/K)$ is defined as the quotient of ${}^2\SE_6(K/F)$ by its centre. 

\section{Some elements of ${}^2\SE_6(K/F)$}

Let $X = (a,b,c\mid A,B,C)$ be an arbitrary element of $\J = \J_K$. First of all we notice that 
the matrices $\delta$ and $\tau$ preserve the Hermitean form, so they encode the elements of 
${}^2\SE_6(K/F)$. 

\begin{lemma}
	\label{lemma:3_isotropic}
	Let $x \in \O_K$ be such that $\ovx^{\s} x = x \ovx^{\s} = 0$. Then $x\ovx = 0$.
\end{lemma}

Consider the matrices
\begin{equation}
	N_x = \begin{bmatrix}
		1 & x & 0 \\
		-\ovx^{\s} & 1 & 0 \\
		0 & 0 & 1
	\end{bmatrix},\quad
	N_x' = \begin{bmatrix}
		1 & 0 & 0 \\
		0 & 1 & x \\
		0 & -\ovx^{\s} & 1
	\end{bmatrix},\quad
	N_x'' = \begin{bmatrix}
		1 & 0 & -\ovx^{\s} \\
		0 & 1 & 0 \\
		x & 0 & 1
	\end{bmatrix},
\end{equation}
where $x\ovx^{\s} = \ovx^{\s}x = 0$. 
It is easy to see that these encode the elements of $\SE_6(K)$. Indeed,
\begin{equation}
	\begin{bmatrix}
		1 & x \\
		0 & 1
	\end{bmatrix}
	\begin{bmatrix}
		1 & 0 \\
		-\ovx ^{\s} & 1
	\end{bmatrix} = 
	\begin{bmatrix}
		1 - x\ovx ^{\s} & x \\
		-\ovx ^{\s} & 1 
	\end{bmatrix} = 
	\begin{bmatrix}
		1 & x \\
		-\ovx ^{\s} & 1
	\end{bmatrix}.
\end{equation}
So, the elements $N_x$, $N_x'$, and $N_x''$ preserve the Dickson--Freudenthal determinant. 
To verify that they preserve $\H$, we look at the action on $\J$:
\begin{equation}
	\label{eq:3_nx_image}
	\begin{array}{r@{\;}c@{\;}l}
		N_x: (a,b,c\mid A,B,C) & \mapsto &
		(a-\Tr(C\ovx ^{\s}), b+\Tr(\ovC x), c \mid \\
		& & \mid	A+\ovx \ovB , B-\ovA \ovx ^{\s},C-x^{\s}\ovC x+ax-bx^{\s}), \\
		
		N_x': (a,b,c\mid A,B,C) & \mapsto & 
		(a, b-\Tr(a\ovx^{\s}), c+\Tr(\ovA x) \mid \\
		& & \mid	A - x^{\s} \ovA x + bx-cx^{\s}, B + \ovx \ovC, C - \ovB \ovx^{\s}), \\
		
		N_x'': (a,b,c\mid A,B,C) & \mapsto &
		(a+\Tr(\ovB x), b, c-\Tr(B\ovx^{\s}) \mid \\
		& & \mid 	A-\ovC\ovx^{\s}, B-x^{\s}\ovB x + cx - ax^{\s}, C+\ovx\ovA).
	\end{array}
\end{equation}
We need to prove an auxiliary lemma.

\begin{lemma}
	\label{lemma:3_oct_aux}
	Suppose that $x,y,z \in \O_F$ with $x \ovx  = 0$. Then
		\begin{enumerate}[(i)]
			\item $x\Tr(yx) = x(yx)$,
			\item $\Tr( (xy)(z\ovx ) ) = 0$.		
		\end{enumerate}
\end{lemma}

\begin{proof}
	\leavevmode
	\begin{enumerate}[(i)]
	
	\item $x\Tr(yx) = x (yx + \ovx \ovy ) = x(yx) + x(\ovx \ovy )
		 = x(yx) + (x\ovx )y = x(yx)$,
		 
	\item $\Tr( (xy)(z\ovx ) ) = \Tr( (z\ovx ) (xy) ) = \Tr( ((z\ovx )x) y ) = 
		\Tr(z(x\ovx )y) = 0$. \qedhere
	\end{enumerate}
\end{proof}

Obviously, it is enough to verify that the elements $N_x$ preserve the Hermitean form $\H$. 
The individual terms in $\H(X) = aa^{\s} + bb^{\s} + cc^{\s} + \Tr(A\ovA ^{\s} + B\ovB ^{\s} +
C\ovC ^{\s})$ are being mapped in the following way:
\begin{equation}
	\begin{array}{r@{\;}c@{\;}l}
		aa^{\s} & \mapsto & aa^{\s} - a^{\s}\Tr(C\ovx ^{\s}) - a\Tr(C^{\s} \ovx ) 
					+ \Tr(C^{\s}\ovx ) \Tr(C\ovx ^{\s}), \\
		bb^{\s} & \mapsto & bb^{\s} + b^{\s}\Tr(\ovC x) + b\Tr(\ovC ^{\s}x^{\s}) +
					\Tr(\ovC x)\Tr(\ovC ^{\s}x^{\s}), \\
		cc^{\s} & \mapsto & cc^{\s}, \\
		\Tr(A\ovA ^{\s}) & \mapsto & \Tr(A\ovA ^{\s}) + \Tr(AB^{\s}x^{\s}) + 
					\Tr(\ovx \ovB \ovA ^{\s}) + 
					\Tr((\ovx \ovB )(B^{\s}x^{\s})), \\
		\Tr(B\ovB ^{\s}) & \mapsto & \Tr(B\ovB ^{\s}) - 
					\Tr(\ovA \ovx ^{\s} \ovB ^{\s}) - \Tr(BxA^{\s}) + 
					\Tr((\ovA \ovx ^{\s})(xA^{\s})), \\
		\Tr(C\ovC ^{\s}) & \mapsto & \Tr(C\ovC ^{\s}) - 
					\Tr(C(\ovx ^{\s}C^{\s}\ovx )) - \Tr(C^{\s}(\ovx C\ovx ^{\s}))
					+ a\Tr(x\ovC ^{\s}) + \\
					& & + a^{\s}\Tr(C\ovx ^{\s}) -
					b^{\s}\Tr(C\ovx ) - b\Tr(x^{\s}\ovC ^{\s}).
	\end{array}
\end{equation}
Using Lemmas \ref{lemma:3_isotropic} and
\ref{lemma:3_oct_aux}, we get
\mbox{$\Tr((\ovA \ovx ^{\s})(xA^{\s})) = 0= \Tr((\ovx \ovB )(B^{\s}x^{\s}))$}.
Next, we also obtain $\Tr(C( \ovx ^{\s} C^{\s} \ovx )) = \Tr(C \ovx ^{\s}
\Tr(C^{\sigma} \ovx )) = \Tr(C^{\sigma} \ovx )\Tr(C\ovx ^{\s})$. Likewise,
we get $\Tr(C^{\s}(\ovx C\ovx ^{\s})) = \Tr ((x^{\s}\ovC x)\ovC ^{\s}) = 
\Tr(\ovC ^{\s} (x^{\s}\ovC x)) = \Tr(\ovC ^{\s} x^{\s})\Tr(\ovC x)$. We see
that all the terms except $aa^{\s},bb^{\s},cc^{\s}$ and $\Tr(A\ovA ^{\s}),
\Tr(B\ovB ^{\s}), \Tr(C\ovC ^{\s})$ cancel out, so it follows that the elements
$N_x$ preserve the Hermitean form. Hence, we have shown the following. 

\begin{proposition}
	The matrices $N_x$, $N_x'$, and $N_x''$ with $x\ovx^{\s} = 0 = \ovx^{\s}x$, encode
	the elements of ${}^2\SE_6(K/F)$. 
\end{proposition}

Of great interest for us is the action of $N_x$ on $\J_{10}^{abC}$. The rest of the section
is devoted to proving the following result.

\begin{theorem}
	The actions of the elements $N_x$ on $\J_{10}^{abC}$, as $x$ ranges through a split
	octonion algebra $\O_K$, generate a group of type $\Omega_{10}^{-,K}(F)$. 
\end{theorem}

We prove this theorem in the series of steps. First, consider the $4$-dimensional 
$K$-subspace $V_4$ of $\J_K$, 
spanned by the Albert vectors of the form $(a,b,0\mid 0,0,C_{-1} e_{-1} + C_1 e_1)$. 

\begin{lemma}
	The actions of the elements $N_{\lambda e_{ \pm 1 }}$ on $V_4$, where $\lambda \in K$, 
	generate a group of type $\Omega_4^{-,K}(F)$. 
\end{lemma}

\begin{proof}
	Consider the basis $\mathcal{B} = \{\ v_1, v_2, v_3, v_4\ \}$ for $V_4$, where
	\begin{equation*}
		\begin{array}{r@{\;}c@{\;}l}
			v_1 & = & (-1,0,0\mid 0,0,0), \\
			v_2 & = & (0,1,0\mid 0,0,0), \\
			v_3 & = & (0,0,0\mid 0,0,-e_{-1}), \\
			v_4 & = & (0,0,0\mid 0,0,e_1).
		\end{array}
	\end{equation*}
	Now we look at the action of $N_{\lambda e_{-1}}$ on these basis elements:
	\begin{equation*}
		\begin{array}{r@{\;}c@{\;}c@{\;}c@{\;}l}
			v_1 & \mapsto & (-1,0,0\mid 0,0,-\lambda e_{-1}) & = & v_1 + \lambda v_3, \\
			v_2 & \mapsto & (0,1,0\mid 0,0,-\lambda^{\s} e_{-1}) & = & v_2 + \lambda^{\s} v_3, \\
			v_3 & \mapsto & (0,0,0\mid 0,0,-e_{-1}) & = & v_3, \\
			v_4 & \mapsto & (-\lambda^{\s}, \lambda, 0 \mid 0,0,-\lambda \lambda^{\s} 
									e_{-1} + e1) & = & 
							-\lambda^{\s} v_1 + \lambda v_2 - \lambda \lambda^{\s} v_3 + v_4.
		\end{array}
	\end{equation*}
	It follows that the element $N_{\lambda e_{-1}}$ can be written as a $4\times 4$ 
	matrix over $\Fqs$ with respect to $\mathcal{B}$:
	\begin{equation*}
		[N_{\lambda e_{-1}}]_{\mathcal{B}} = \begin{bmatrix}
			1 & 0 & \lambda & 0 \\
			0 & 1 & \lambda^{\s} & 0 \\
			0 & 0 & 1 & 0 \\
			\lambda^{\s} & \lambda & \lambda \lambda^{\s} & 1
		\end{bmatrix}.
	\end{equation*}
	For convenience, instead of the element $N_{\lambda e_1}$ we consider the element
	$N_{-\lambda^{\s} e_1}$ which acts on the elements of $\mathcal{B}$ as follows:
	\begin{equation*}
		\begin{array}{r@{\;}c@{\;}c@{\;}c@{\;}l}
			v_1 & \mapsto & (-1,0,0\mid 0,0, \lambda^{\s} e_1) & = & v_1 + \lambda^{\s} v_4, \\
			v_2 & \mapsto & (0,1,0\mid 0,0, \lambda e_1) & = & v_2 + \lambda v_4, \\
			v_3 & \mapsto & (-\lambda, \lambda^{\s}, 0 \mid 0,0,-e_{-1}+\lambda \lambda^{\s} 
							e_1) & = & 
								\lambda v_1 + \lambda^{\s} v_2 + v_3 + \lambda \lambda^{\s} v_4, \\
			v_4 & \mapsto & (0,0,0\mid 0,0,e_1) & = & v_4.
		\end{array}
	\end{equation*}
	The matrix $[N_{-\lambda^{\s} e_1}]_{\mathcal{B}}$ has the form
	\begin{equation*}
		[N_{-\lambda^{\s} e_1}]_{\mathcal{B}} = \begin{bmatrix}
			1 & 0 & 0 & \lambda^{\s} \\
			0 & 1 & 0 & \lambda \\
			\lambda & \lambda^{\s} & 1 & \lambda \lambda^{\s} \\
			0 & 0 & 0 & 1
		\end{bmatrix}.
	\end{equation*}
	Consider the basis $\mathcal{B}' = \{\ v_3, v_4, v_1 v_4\ \}$ obtained as a permutation
	of the elements in $\mathcal{B}$. With respect to $\mathcal{B}'$ the $4\times 4$
	 matrices
	$[N_{\lambda e_{-1}}]_{\mathcal{B'}}$ and 
	$[N_{-\lambda^{\s} e_1}]_{\mathcal{B}'}$ take the form
	\begin{equation*}
		[N_{\lambda e_{-1}}]_{\mathcal{B'}}	= \begin{bmatrix}
			1 & 0 & 0 & 0 \\
			\lambda^{\s} & 1 & 0 & 0 \\
			\lambda & 0 & 1 & 0 \\
			\lambda\lambda^{\s} & \lambda & \lambda^{\s} & 1
		\end{bmatrix},\quad 
		[N_{-\lambda^{\s} e_1}]_{\mathcal{B}'} = \begin{bmatrix}
			1 & \lambda^{\s} & \lambda & \lambda\lambda^{\s} \\
			0 & 1 & 0 & \lambda \\
			0 & 0 & 1 & \lambda^{\s} \\
			0 & 0 & 0 & 1
		\end{bmatrix}.
	\end{equation*}
	We notice that 
	\begin{equation*}
		[N_{\lambda e_{-1}}]_{\mathcal{B'}} = \begin{bmatrix}
			1 & 0 \\
			\lambda & 1 
		\end{bmatrix} \otimes
		\begin{bmatrix}
			1 & 0 \\
			\lambda^{\s} & 1
		\end{bmatrix},\quad
		[N_{-\lambda^{\s} e_1}]_{\mathcal{B}'} = \begin{bmatrix}
			1 & \lambda \\
			0 & 1
		\end{bmatrix} \otimes
		\begin{bmatrix}
			1 & \lambda^{\s} \\
			0 & 1
		\end{bmatrix},
	\end{equation*}
	where $\otimes$ is the Kronecker product of two matrices. The mapping
	\begin{equation*}
		\begin{bmatrix}
			1 & 0 \\
			\lambda & 1 
		\end{bmatrix} \mapsto
		\begin{bmatrix}
			1 & 0 \\
			\lambda & 1
		\end{bmatrix} \otimes
		\begin{bmatrix}
			1 & 0 \\
			\lambda^{\s} & 1
		\end{bmatrix},\quad 
		\begin{bmatrix}
			1 & \lambda \\
			0 & 1
		\end{bmatrix} \mapsto
		\begin{bmatrix}
			1 & \lambda \\
			0 & 1
		\end{bmatrix} \otimes
		\begin{bmatrix}
			1 & \lambda^{\s} \\
			0 & 1
		\end{bmatrix}
	\end{equation*}
	can be extended to a homomorphism $\phi$ which is obviously surjective as $\lambda$
	 ranges
	through the whole field $K$. Its kernel is a subgroup
	\begin{equation*}
		\ker(\phi) = 
		\left\langle
			\begin{bmatrix}
				-1 & 0 \\
				0 & -1
			\end{bmatrix}
		\right\rangle
	\end{equation*}
	which has order $2$, so we get the action of the group $\PSL_2(K)$ on $V_4$ since
	the matrices
	\begin{equation*}
		\begin{bmatrix}
			1 & 0 \\
			\lambda & 1
		\end{bmatrix},\quad
		\begin{bmatrix}
			1 & \lambda \\
			0 & 1 
		\end{bmatrix}
	\end{equation*}
	generate a group $\SL_2(K)$ by a well-known result.
	Therefore, as $\PSL_2(K) \cong \Omega_4^-(q)$ (\ref{theorem:L2Omega4}), 
	we have the action of $\Omega_4^{-,K}(F)$. 
\end{proof}

We again use the results of section \ref{section:orthogonal}. Consider the $6$-dimensional
$K$-subspace $V_6$ spanned by the Albert vectors of the form 
$(a,b,0\mid 0,0,C)$ with $C \in \langle e_{-1}, e_{\bar{\omega}}, e_{-\bar{\omega}}, e_1 \rangle$.
Our copy of $\Omega_4^{-,K}(F)$ preserves two isotropic Albert Vectors in $V_6$: 
\begin{equation}
	\begin{array}{r@{\;}c@{\;}l}
		u_{\bar{\omega}} & = & (0,0,0\mid 0,0,e_{\bar{\omega}}), \\
		u_{-\bar{\omega}} & = & (0,0,0\mid 0,0,e_{-\bar{\omega}}). \\
	\end{array} 
\end{equation}
The element $N_{e_{\bar{\omega}}}$ preserves $u_{\bar{\omega}}$, but not $u_{-\bar{\omega}}$. 
Therefore, adjoining its action to our $\Omega_4^{-,K}(F)$, we obtain a subgroup of 
$V_4\cn\Omega_4^{-,K}(F)$. We know that $\Omega_4^{-,K}(F)$ is maximal in the latter, so 
we conclude that the actions of $N_{\lambda e_{\pm 1}}$ and $N_{e_{\bar{\omega}}}$ on 
$V_6$ is that of $V_4\cn\Omega_4^{-,K}(F)$. Next, the element $N_{e_{-\bar{\omega}}}$
preserves $V_6$ but it does not preserve $\langle u_{\bar{\omega}} \rangle$, so by Theorem
\ref{theorem:1_space_stab}, adjoining $N_{e_{-\bar{\omega}}}$ to $V_4\cn\Omega_4^{-,K}(F)$,
we get the action of $\Omega_6^{-,K}(F)$ on $V_6$.

Next, take the $8$-space $V_8$ spanned by the Albert vectors $(a,b,0\mid 0,0,C)$ with 
$C \in \langle e_{-1}, e_{\bar{\omega}}, e_{\omega}, 
e_{- \omega}, e_{-\bar{\omega}}, e_1 \rangle$, and consider two isotropic vectors
\begin{equation}
	\begin{array}{r@{\;}c@{\;}l}
		u_{\omega} & = & (0,0,0 \mid 0,0,e_{\omega}), \\
		u_{-\omega} & = & (0,0,0 \mid 0,0,e_{-\omega}),
	\end{array}
\end{equation}
which are fixed by our copy of $\Omega_6^{-,K}(F)$. The action of $N_{e_{\omega}}$ on $V_8$
preserves $u_{\omega}$ but not $u_{-\omega}$, and therefore adjoining this element to 
$\Omega_6^{-,K}(F)$ we get the action of the group $V_6\cn\Omega_6^{-,K}(F)$. The element
$N_{e_{-\omega}}$ does not preserve the $1$-space $\langle u_{\omega} \rangle$, so appending it to 
$V_6\cn\Omega_6^{-,K}(F)$, we get the action of $\Omega_8^{-,K}(F)$ on $V_8$. 

Finally, we choose two isotropic Albert vectors
\begin{equation}
	\begin{array}{r@{\;}c@{\;}l}
		u_0 & = & (0,0,0\mid 0,0,e_0), \\
		u_{-0} & = & (0,0,0\mid 0,0,e_{-0}).
	\end{array}
\end{equation}
in $\J_{10}^{abC}$. We adjoin the element $N_{e_{0}}$ which fixes $u_0$ but not $u_{-0}$ to 
get the action of the group $V_8\cn \Omega_8^{-,K}(F)$. Appending to this the action of
$N_{e_{-0}}$ which does not preserve $\langle u_0 \rangle$, we obtain the action of
$\Omega_{10}^{-,K}(F)$.

\section{Action of ${}^2\SE_6(K/F)$ on white points}

As in the case of $\SE_6$, we are interested in the action on white points. We will, however, see 
that although $SE_6$ acts transitively on white points, the action of ${}^2\SE_6(K/F)$ splits
into several orbits. 

We first consider some examples. Suppose $W_1 = (0,0,0\mid 0,0,e_0)$. As we know (\ref{eq:17_space}), 
it determines 
a $17$-space $	\{\ (a,b,0 \mid A,B,C)\ \big|\ e_0 A = B e_0 = \Tr(e_0 \ovC) = 0\ \}$. A
straightforward calculation shows that
this $17$-space $U_1$ is spanned by the Albert vectors of the form \mbox{$(a,b,0\mid A,B,C)$} with
\begin{equation}
	\begin{array}{r@{\;}c@{\;}l}
		A & \in & \langle e_{\bar{\omega}}, e_{\omega}, e_{-0}, e_1 \rangle, \\
		B & \in & \langle e_{-1}, e_{-0}, e_{-\omega}, e_{-\bar{\omega}} \rangle, \\
		C & \in & \langle e_{-1}, e_{\bar{\omega}}, e_{\omega}, e_0, e_{-\omega}, e_{-\bar{\omega}}, e_1
		\rangle.
	\end{array}
\end{equation}
We are also interested in the radical $R_1$ of $\H$ inside this $17$-space. In our case it is spannned
by the vectors of the form \mbox{$(0,0,0\mid A,B,C)$} with
\begin{equation}
	\begin{array}{r@{\;}c@{\;}l}
		A & \in & \langle e_{\bar{\omega}}, e_{\omega}, e_{-0}, e_1 \rangle, \\
		B & \in & \langle e_{-1}, e_{-0}, e_{-\omega}, e_{-\bar{\omega}} \rangle, \\
		C & \in & \langle e_0 \rangle.
	\end{array}
\end{equation}
In other words, our vector $W_1$ determines the $17$-space $U_1$ and the $9$-dimensional radical $R_1$ 
of $\H$ in $U_1$. Note that $W_1$ is isotropic with respect to $\H$, and also $W_1 \in R_1$. 

It turns out that there is another type of isotropic white vectors. Consider 
$W_2 = (0,0,0 \mid 0,0,e_0 + \lambda e_1)$, where $\lambda \in K\setminus F$. It again determines a 
$17$-space $U_2$, spanned by the Albert vectors of the form $(a,b,0\mid A,B,C)$ with 
\begin{equation}
	\begin{array}{r@{\;}c@{\;}l}
		A & \in & \langle e_{\bar{\omega}}+\lambda e_{-\omega}, e_{\omega}-\lambda e_{-\bar{\omega}}
		, e_{-0}, e_1
		\rangle, \\
		B & \in & \langle e_{-1}+\lambda e_0, e_{-0} - \lambda e_1,
	e_{-\omega}, e_{-\bar{\omega}}  \rangle, \\
		C & \in & \langle e_{-1}-\lambda^2 e_1 - \lambda, e_{\bar{\omega}}, e_{\omega},
	e_0 + \lambda e_1, e_{-\omega}, e_{-\bar{\omega}}, e_1 \rangle.
	\end{array}
\end{equation}
We find that the radical $R_2$ of $\H$ in $U_2$ is spanned by the vectors of the form 
$(0,0,0\mid A,B,C)$ with 
\begin{equation}
	\begin{array}{r@{\;}c@{\;}l}
		A & \in & \langle e_{-0}, e_1 \rangle, \\
		B & \in & \langle e_{-\omega}, e_{-\bar{\omega}} \rangle, \\
		C & \in & \langle e_0 + \lambda^q e_1 \rangle,
	\end{array}
\end{equation}
i.e. it is $5$-dimensional. We notice that $W_2$ is isotropic, but in this case $W_2 \not\in R_2$.

We conclude that the white points $\langle W_1 \rangle$ and $\langle W_2 \rangle$ belong to different
orbuts under the action of ${}^2\SE_6(K/F)$. Of course, there is also at least one orbit on the 
non-isotropic white points. 

\subsection{Orbits of ${}^2\SE_6^K(F)$ on white points}
\subsection{The stabiliser of type 1 vector}
\subsection{The stabiliser of type 2 vector}
\subsection{The stabiliser of type 3 vector}

Consider $W_3 = (0,0,1 \mid 0,0,0)$. We prove the following theorem.

\begin{theorem}
	The stabiliser in ${}^2\SE_6(K/F)$ of $W_3$ is the subgroup of shape $\Spin_{10}^{-,K}(F)$. 
\end{theorem}

\begin{proof}
	
\end{proof}

\section{Case of a finite field}
\subsection{White vectors in $\J_8^C$}

As a practical counting excersise, we count the isotropic white vectors in $\J_{8}^{C}$.
As before, $K = \Fqs$. First, we need the following auxiliary result.

\begin{lemma}
		\label{lemma:zn}
		Let $V$ be a vector space over $\Fqs$ of dimension $2m$. 
		Define the map $Z_m: V \rightarrow \Fq$ in the following way:
		\begin{equation*}
			Z_m(x) = (x_1^q - x_1) (x_2^q - x_2) + (x_3^q - x_3) (x_4^q - x_4) + \cdots +
			(x_{2m-1}^q-x_{2m-1}) (x_{2m}^q - x_{2m}),
		\end{equation*}
		where $x = (x_1, ..., x_{2m})$. 
		Denote by $z_m$ the number of $x\in V$ such that $Z_m(x) = 0$. Then
		\begin{equation*}
			z_n = q^{3m-1}(q^m+q-1).
		\end{equation*}
\end{lemma}

\begin{proof}
		We proceed by induction on $m$. If $m=1$, the equality $Z_m(x)=0$ reduces to
		\begin{equation*}
			(x_1^q - x_1) (x_2^q - x_2) = 0.
		\end{equation*}
		Note that this is possible when $x_1^q = x_1$ or $x_2^q = x_2$, i.e. when
		$x_1 \in \Fq$ or $x_2 \in \Fq$. Thus, when $m=1$ there are precisely
		$2q^3-q^2 = q^2(q+q-1)$ solutions.

		Now suppose that the statement holds for all integers $k$ such that
		$1 \leqslant k \leqslant m-1$. In the case
		\begin{equation*}
			\left.
			\begin{array}{l}
				(x_1^q - x_1) (x_2^q - x_2) = 0, \\
				(x_3^q - x_3) (x_4^q - x_4) + \cdots + 
					(x_{2m-1}^q-x_{2m-1}) (x_{2m}^q - x_{2m}) = 0,
			\end{array}
			\right\}
		\end{equation*}
		we get $z_1 z_{m-1}$ solutions. On the other hand, if
		\begin{equation*}
			\left.
			\begin{array}{l}
				(x_1^q - x_1) (x_2^q - x_2) = \lambda, \\
				(x_3^q - x_3) (x_4^q - x_4) + \cdots + (x_{2m-1}^q-x_{2m-1})
						(x_{2m}^q - x_{2m}) = -\lambda
			\end{array}
			\right\}
		\end{equation*}
		for $0\neq \lambda \in \Fq$, there are 
		\begin{equation*}
			(q^4-z_1) \frac{(q^{4(m-1)}-z_{m-1})}{q-1}
		\end{equation*}
		solutions.
		We divide the second factor by $(q-1)$ since each pair $(x_1,x_2)$ satisfying the
		first equation, fixes the value of $\lambda$ for the second equation. Overall we have
		\begin{equation*}
			z_m = z_1 z_{m-1} + (q^4-z_1) \frac{(q^{4(m-1)}-z_{m-1})}{q-1}.
		\end{equation*}
		Thus, we have obtained a recurrence relation and by substituting $z_1$ and $z_{m-1}$ in
		it, we finally obtain $z_m = q^{3m-1}(q^m+q-1)$.
\end{proof}

The following theorem allows us to count the elements of $V$ satisfying simultaneously a 
certain quadratic and a certain Hermitean form.

	\begin{theorem}
		Let $V$ be an vector space over $\Fqs$ of dimension $2m$. 
		Let the quadratic form $Q_m:V \rightarrow \Fqs$ be defined as
		\begin{equation}
			Q_m(x) = x_1 x_2 + x_3 x_4 + \cdots + x_{2m-1} x_{2m},
		\end{equation}
		where $x = (x_1, ..., x_{2m})$, 
		and also define the Hermitean form $H_m:V \rightarrow \Fq$ by
		\begin{equation}
			H_m(x) = x_1^q x_2 + x_1 x_2^q + x_3^q x_4 + x_3 x_4^q + \cdots + x_{2m-1}^q x_{2m}
				+ x_{2m-1} x_{2m}^q.
		\end{equation}
		Let $w_m$ be the number of $x \in V$ such that
		\begin{equation}
			\label{eq:system1}
			\left.
				\begin{array}{l}
					Q_m(x) = 0, \\
					H_m(x) = 0.
				\end{array}
			\right\}
		\end{equation}
		Then
		\begin{equation}
			w_m = q^{2m} + q^{2m-1}(q^m-1)(q^{m-2}+1).
		\end{equation}
	\end{theorem}

	\begin{proof}
		We again proceed by induction on $m$. When $m=1$, the system (\ref{eq:system1}) reduces to
		\begin{equation*}
			\left.
				\begin{array}{l}
					x_1 x_2 = 0, \\
					x_1^q x_2 + x_1 x_2^q = 0.
				\end{array}
			\right\}
		\end{equation*}
		Note that each pair $(x_1,x_2)$ which satisfies the first equation also satisfies the second
		one, so in this case the number of solutions is $2q^2-1 = q^2 + q(q-1)(q^{-1}+1)$.

		Suppose now that the statement holds for all integers $k$ such that
		$1 \leqslant k \leqslant m-1$ and consider the following system:
		\begin{equation*}
			\left.
				\begin{array}{l}
					x_1 x_2 + x_3 x_4 + \cdots + x_{2m-1} x_{2m} = 0, \\
					x_1^q x_2 + x_1 x_2^q + x_3^q x_4 + x_3 x_4^q + \cdots + x_{2m-1}^q x_{2m}
				+ x_{2m-1} x_{2m}^q = 0.
				\end{array}
			\right\}
		\end{equation*}
		We distinguish two cases.

		First, consider the case $x_1 = 0$. Then $x_2$ can take any of the $q^2$ possible values
		and the remaining system is equivalent to
		\begin{equation*}
			\left.
				\begin{array}{l}
					Q_{m-1}(x) = 0, \\
					H_{m-1}(x) = 0,
				\end{array}
			\right\}
		\end{equation*}
		so there are $q^2 w_{m-1}$ solutions in this case.

		Now suppose that $x_1 \neq 0$. Without loss of generality we may consider the case $x_1=1$.
		The system (\ref{eq:system1}) takes the form
		\begin{equation*}
			\left.
				\begin{array}{l}
					x_2 = - x_3 x_4 - \cdots - x_{2m-1} x_{2m}, \\
					x_2 + x_2^q + x_3^q x_4 + x_3 x_4^q + \cdots + x_{2m-1}^q x_{2m}
				+ x_{2m-1} x_{2m}^q = 0.
				\end{array}
			\right\}
		\end{equation*}
		We substitute $x_2$ from the first equation into the second one to obtain
		\begin{equation*}
			(x_3^q-x_3)(x_4^q-x_4) + \cdots + (x_{2m-1}^q-x_{2m-1})(x_{2m}^q-x_{2m}) = 0.
		\end{equation*}
		Using the result of Lemma \ref{lemma:zn}, we obtain that in this case there are
		$(q^2-1)z_{m-1}$ solutions. In total, we obtain the following recurrence relation:
		\begin{equation*}
			w_n = q^2 w_{m-1} + (q^2-1) z_{m-1}.
		\end{equation*}
		By substituting the appropriate values for $w_{m-1}$ and $z_{m-1}$, we obtain the result.
	\end{proof}

An Albert vector $(0,0,0\mid 0,0,C) \in \J_8^C$ is white if and only if $C\ovC = 0$. Recall that
$\O_K$ is split, so we can use our favourite basis $\{\ e_i\ \big|\ i \in \pm I\ \}$. Note that
with respect to this basis $C\ovC = 0$ is equivalent to 
\begin{equation}
	C_{-1}C_{1} + C_{\bar{\omega}} C_{-\bar{\omega}} + C_{\omega} C_{-\omega} + C_{-0} C_{0} = 0.
\end{equation}
Next, $(0,0,0\mid 0,0,C)$ is isotropic if and only if $\Tr(C \ovC^{\s}) = 0$, which is equivalent
to
\begin{equation}
	C_{-1}^qC_{1} + C_{-1}C_{1}^q + 
	C_{\bar{\omega}}^q C_{-\bar{\omega}} + C_{\bar{\omega}} C_{-\bar{\omega}}^q + 
	C_{\omega}^q C_{-\omega} + C_{\omega} C_{-\omega}^q + 
	C_{-0}^q C_{0} + C_{-0} C_{0}^q = 0.
\end{equation}

We know that there are exactly $(q^8-1)(q^6+1)$ white vectors in $\J_8^C$. Furthermore,
there are
\begin{equation}
	w_4 - 1 = (q^2+1)(q^3+1)(q^3(q^2+1)(q-1)+(q^5+1))
\end{equation}
isotropic white vectors of the form $(0,0,0 \mid 0,0,C)$ and
\begin{equation}
q^6(q^4-1)(q^3-1)(q-1)
\end{equation}
non-isotropic white vectors of the same form.

Now, using Proposition \ref{prop:3_2forms_orbits}, we find that the full subgroup of 
$\SE_6(K)$ which preserves $\QQ$ 
and
$\H$ on $\J_8^C$ has three orbits on white points, and the sizes of these orbits are given by
\begin{enumerate}[(i)]
	\item $(q^4-1)(q^3+1)/(q-1)$;
	\item $q(q^6-1)(q^4-1)(q^2+1)/(q^2-1)$;
	\item $q^6 (q^4-1)(q^3-1)/(q+1)$.
\end{enumerate}
Of these, the first two are isotropic, while the last one is non-isotropic. 

\subsection{White vectors in $\J_{16}^{BC}$}

We can also count the isotropic white vectors in $\J_{16}^{BC}$, $K = \Fqs$. Suppose 
\mbox{$X = (0,0,0\mid 0,B,C)$} is white and note that the whiteness conditions 
take form $B\ovB = C\ovC = 0 = BC$. 

First, we count the white vectors $(0,0,0\mid 0,B,C)$ such that $B \neq 0$ and $C \neq 0$.
We notice that given $B \neq 0$ and $B\ovB = 0 = BC$, we automatically have $C\ovC = 0$. 
Indeed, for if $C\ovC \neq 0$, $C$ is invertible and $BC = 0$ implies $B = 0$, a contradiction.
So, there are $(q^8-1)(q^6+1)$ choices for $B$ and $(q^8-1)$ choices for $C$ 
(see Lemma \ref{prop:octonion_annihilator}). In total, there are $(q^8-1)^2 (q^6+1)$ white
vectors with $B \neq 0$ and $C \neq 0$. 

To count the isotropic white vectors of the form $(0,0,0\mid 0,B,C)$ we distiguish two cases:
\begin{equation}
	\label{eq:8twocases}
	\left.
		\begin{array}{l}
			B\ovB = BC = 0,\\
			\Tr(B\ovB^{\s}) = 0,\\
			\Tr(C\ovC^{\s}) = 0,\\
		\end{array}
	\right\},\quad
	\left.
		\begin{array}{l}
			B\ovB = BC = 0,\\
			\Tr(B\ovB^{\s}) = \lambda \neq 0,\\
			\Tr(C\ovC^{\s}) = -\lambda.\\
		\end{array}
	\right\}.
\end{equation}

In the previous section we learned that the subgroup of $\SE_6(K)$, preserving $\QQ$ and $\H$
on $\J_8^C$ has two orbits on isotropic white points. Proposition \ref{prop:3_2forms_orbits}
the first orbit consists of white points
$\langle X \rangle$ where $X$ is written over $\Fq$. That is, its representatives are
one-dimensional $\Fqs$-subspaces generated by the white vectors written over $\Fq$.
Suppose $X_C = (0,0,0\mid 0,0,C)$ belongs to the first orbit. By taking a particular candidate 
for $C$, it is easy to see that in this case there are $q^8-1$ choices for $B$.
Now, if $X_C$ belongs to the second orbit, there are $q^4(q^3+q^2-q)-1$ choices for $B$.
If, on the other hand, $X_C$ is non-isotropic, then there are $q^3(q^4-1)$ choices for $B$, and overall
we have 
\begin{multline}
	(q^4-1)(q^3+1)(q+1)(q^8-1) \\
	+\ q(q^6-1)(q^4-1)(q^2+1)(q^4(q^3+q^2-q)-1) \\
	+\ q^6(q^4-1)(q^3-1)(q-1)q^3(q^4-1) \\
	=\ (q^{13}+q^{11}-q^{10}+2q^9+q^8+2q^7-q^6+2q^4+1)(q^4-1)^2
\end{multline}
isotropic white vectors of the form $(0,0,0\mid 0,B,C)$ with $B,C \neq 0$. 

Recall that the group preserving $Q_4$ and $H_4$ has two orbits on the isotropic white
points in $J_8$ with
$(q^4-1)(q^3+1)/(q-1)$ and $q(q^6-1)(q^4-1)(q^2+1)/(q^2-1)$ elements. Again, by PropositionThe first orbit
consists of white points
$\langle X \rangle$ where $X$ is written over $\Fq$. That is, its representatives are
one-dimensional $\Fqs$-subspaces generated by the white vectors written over $\Fq$.
Since we know the totality of white vectors of this form, we find that there are
\begin{equation}
	q^6(q^4-1)^2(q^4+2)(q^3-1)(q-1)
\end{equation}
non-isotropic white vectors in $\J_{16}^{BC}$ with $B,C\neq 0$. Overall there are
\begin{multline}
	q^6(q^4-1)^2(q^4+2)(q^3-1)(q-1)\\
	+\ 2q^6(q^4-1)(q^3-1)(q-1)\\
	=\ q^{10}(q^8-1)(q^3-1)(q-1)
\end{multline}
non-isotropic white vectors in $\J_{16}^{BC}$.