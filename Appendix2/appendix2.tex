\chapter{B: $\Omega_4^+(F)$}
\label{AppB}
\stepcounter{appendixsection}


In this appendix we prove a basic result related to the group
$\Omega_4^+(F)$. Let $V_4$ be a $4$-dimensional vector space over $F$ and 
let $\mathcal{B} = (v_1, v_2, v_3, v_4)$ be its basis:
\begin{equation}
	\begin{array}{r@{\;}c@{\;}l}
		v_1 & = & [1,0,0,0], \\
		v_2 & = & [0,1,0,0], \\
		v_3 & = & [0,0,1,0], \\
		v_4 & = & [0,0,0,1].
	\end{array}
\end{equation}
Define the quadratic 
form $\QQ_4$ of plus type via
\begin{equation}
	\QQ_4([a, b, c, d]) = ad - bc.
\end{equation} 

\begin{lemma}
	\label{lemma:B_omega4plus}
	The matrices
	\begin{multline}
		A_{\lambda}=\begin{bmatrix}
			1 & 0 & \lambda & 0 \\
			0 & 1 & 0 & \lambda \\
			0 & 0 & 1 & 0 \\
			0 & 0 & 0 & 1
		\end{bmatrix},\ 
		B_{\lambda}=\begin{bmatrix}
			1 & \lambda & 0 & 0 \\
			0 & 1 & 0 & 0 \\
			0 & 0 & 1 & \lambda \\
			0 & 0 & 0 & 1
		\end{bmatrix}, \\
		C_{\lambda}=\begin{bmatrix}
			1 & 0 & 0 & 0 \\
			\lambda & 1 & 0 & 0 \\
			0 & 0 & 1 & 0 \\
			0 & 0 & \lambda & 1
		\end{bmatrix},\  
		D_{\lambda}=\begin{bmatrix}
			1 & 0 & 0 & 0 \\
			0 & 1 & 0 & 0 \\
			\lambda & 0 & 1 & 0 \\
			0 & \lambda & 0 & 1
		\end{bmatrix}
	\end{multline}
	generate the group $\Omega_4^+(F)$ as $\lambda$ ranges through $F$.
\end{lemma}

\begin{proof}
	We notice
	\begin{multline*}
		A_{\lambda} = 
		\begin{bmatrix}
			1 & \lambda \\
			0 & 1
		\end{bmatrix} \otimes
		\begin{bmatrix}
			1 & 0 \\
			0 & 1
		\end{bmatrix},\ \ 
		B_{\lambda} = 
		\begin{bmatrix}
			1 & 0 \\
			0 & 1
		\end{bmatrix} \otimes
		\begin{bmatrix}
			1 & \lambda \\
			0 & 1 
		\end{bmatrix}, \\
		C_{\lambda}=\begin{bmatrix}
			1 & 0 \\
			0 & 1
		\end{bmatrix} \otimes
		\begin{bmatrix}
			1 & 0 \\
			\lambda & 1
		\end{bmatrix},\ \ 
		D_{\lambda}=  
		\begin{bmatrix}
			1 & 0 \\
			\lambda & 1
		\end{bmatrix} \otimes
		\begin{bmatrix}
			1 & 0 \\
			0 & 1
		\end{bmatrix},
	\end{multline*}	
	where $\otimes$ is the Kronecker product. As we know, 
	\begin{equation*}
		\left\langle
			\begin{bmatrix}
				1 & \lambda \\
				0 & 1
			\end{bmatrix},\ 
			\begin{bmatrix}
				1 & 0 \\
				\lambda & 1
			\end{bmatrix}\ 
			\bigg|
			\ 
			\lambda \in F
		\right\rangle \cong \SL_2(F).
	\end{equation*}
	It follows that
	\begin{multline*}
		\left\langle
			\begin{bmatrix}
				1 & \lambda \\
				0 & 1
			\end{bmatrix} \otimes
			\begin{bmatrix}
				1 & 0 \\
				0 & 1
			\end{bmatrix},\ 
			\begin{bmatrix}
				1 & 0 \\
				\lambda & 1
			\end{bmatrix}\otimes
			\begin{bmatrix}
				1 & 0 \\
				0 & 1
			\end{bmatrix}\ 
			\bigg|
			\ 
			\lambda \in F
		\right\rangle \cong \\
		\cong 
		\left\langle
			\begin{bmatrix}
				1 & 0 \\
				0 & 1
			\end{bmatrix} \otimes
			\begin{bmatrix}
				1 & \lambda \\
				0 & 1
			\end{bmatrix},\ 
			\begin{bmatrix}
				1 & 0 \\
				0 & 1
			\end{bmatrix} \otimes
			\begin{bmatrix}
				1 & 0 \\
				\lambda & 1
			\end{bmatrix}\ 
			\bigg|
			\ 
			\lambda \in F
		\right\rangle \cong \SL_2(F).
	\end{multline*}
	These two copies of $\SL_2(F)$ clearly commute, thus
	their intersection is contained in the centre of each copy, 
	which means that it can only contain $\pm \II_2$.
	Since
	\begin{equation*}
		\begin{bmatrix}[r]
				-1 & 0 \\
				0 & -1
		\end{bmatrix} \otimes
		\begin{bmatrix}[r]
				-1 & 0 \\
				0 & -1
		\end{bmatrix} = 
		\begin{bmatrix}
			1 & 0 & 0 & 0 \\
			0 & 1 & 0 & 0 \\
			0 & 0 & 1 & 0 \\
			0 & 0 & 0 & 1
		\end{bmatrix},
	\end{equation*}
	we finally get
	\begin{multline*}
		\left\langle
			\begin{bmatrix}
				1 & 0 \\
				0 & 1
			\end{bmatrix} \otimes 
			\begin{bmatrix}
				1 & \lambda \\
				0 & 1
			\end{bmatrix},\ 
			\begin{bmatrix}
				1 & 0 \\
				0 & 1
			\end{bmatrix} \otimes
			\begin{bmatrix}
				1 & 0 \\
				\lambda & 1
			\end{bmatrix}, \right. \\
			\left. \begin{bmatrix}
				1 & \lambda \\
				0 & 1
			\end{bmatrix} \otimes 
			\begin{bmatrix}
				1 & 0 \\
				0 & 1
			\end{bmatrix},\ 
			\begin{bmatrix}
				1 & 0 \\
				\lambda & 1
			\end{bmatrix} \otimes
			\begin{bmatrix}
				1 & 0 \\
				0 & 1
			\end{bmatrix}\ 
			\bigg|
			\ 
			\lambda \in F
		\right\rangle \cong \SL_2(F) \circ \SL_2(F).
	\end{multline*}
	Now, $\SL_2(F) \circ \SL_2(F) \cong \Omega_4^+(F)$, and what left is to show that those matrices which generate 
	$\SL_2(F) \circ \SL_2(F)$ also generate the whole of $\Omega_4^+(F)$. Indeed, if $F$ is infinite, having just an isomorphism
	is not sufficient, since in this case the group can easily be isomorphic to a proper subgroup of itself. 
	Indeed, for example, $(\mathbb{Z}, +) \cong (2\mathbb{Z}, +)$.
First, we show that the action
	of the group $G = \SL_2(F) \circ \SL_2(F) \leqslant \Omega_4^+(F)$ on the isotropic vectors in $V_4$ is transitive. 
	
	Let $v = [a,\alpha,\beta,b]$ be an arbitrary isotropic vector in $V_4$.
	We find $Q(v) = ab - \alpha \beta$, so $v$ is isotropic if and only if $(a,b,\alpha,\beta) \neq
	(0,0,0,0)$ and $ab = \alpha \beta$. 
	
	First, consider the case $(a, b) = (0,0)$. For $v$ to be isotropic, it has to be either 
	\mbox{$\alpha = 0, \beta \neq 0$} or 
	$\alpha \neq 0, \beta = 0$. Acting on $[0,0,\beta,0]$ by $C_{\beta^{-1}}$, 
	we obtain the vector
	$[1,0,\beta,0]$. Next, we act by $A_{-\beta}$ to obtain $[1,0,0,0]$. Similarly,
	acting on $[0,\alpha,0,0]$ with $\alpha \neq 0$ by $D_{\alpha^{-1}}$, and then by $B_{-\alpha}$, 
	we get the vector $[1,0,0,0]$ as well.
	
	If $a \neq 0$ and $b \neq 0$, we are forced to have $\alpha \neq 0$ and $\beta \neq 0$. Acting by 
	$A_{b\alpha^{-1}}$, 
	we obtain the vector $[a,\alpha,\beta+b\alpha^{-1},0]$. 
	Next, we consider the vector \mbox{$[a,\alpha,\beta,0]$}
	with $a \neq 0$. If $\alpha \neq 0$, then we may act by $D_{-a\alpha^{-1}}$ to obtain 
	$[0,\alpha,\beta,0]$; on the other hand,
	if $\beta \neq 0$, we may act by $C_{-a\beta^{-1}}$ to obtain the same vector. If $(\alpha, \beta) = (0,0)$, then we map 
	$[a,0,0,0]$ to $[a,0,1,0]$ by the action of $B_{a^{-1}}$, and further to $[0,0,1,0]$ by the action of 
	$C_{-a}$. 
	In other words, we have reduced to the case $(a, b) = (0, 0)$. 
	
	Similarly, if $a = 0$ and $b \neq 0$, we consider $[0,\alpha,\beta,b]$, and act on it by 
	$C_{-b\alpha^{-1}}$ if $\alpha \neq 0$
	or by $D_{-b\beta^{-1}}$ if $\beta \neq 0$ to obtain $[0,\alpha,\beta,0]$. 
	Finally, if $(\alpha, \beta) = (0, 0)$, then
	we act on $[0,0,0,b]$ by $D_{b^{-1}}$, followed by the action of $A_b$ to obtain $(0,0,0 \mid 0,0,e_1)$ which again 
	brings us back to the case $(a, b) = (0,0)$, and we are done proving transitivity of $G$ on the isotropic vectors in $V_4$.
	
	The rest is to show that the matrices $A_{\lambda}$,$B_{\lambda}$,$C_{\lambda}$, and $D_{\lambda}$ indeed generate 
	the whole $\Omega_{4}^+(F)$. Let $g$ be an element of $\Omega_4^+(F)$. Since now we know that 
\mbox{$\SL_2(F) \circ \SL_2(F)$} generated by
	these matrices is transitive on isotropic vectors in $V_4$, we may choose an element $h_1 \in \SL_2(F) \circ \SL_2(F)$ such that 
	$g h_1^{-1}$ stabilises $v_1$. We let $w = v_2^{g h_1^{-1}}$ and write down the conditions on $w$:
	\begin{equation*}
		\left.
		\begin{array}{r@{\;}c@{\;}c@{\;}c@{\;}l}
			\inner{v_1}{w} & = & \inner{v_1}{v_2} & = & 0, \\
			\QQ_4(w) & = & \QQ_4(v_2) & = & 0, \\
			\dim_F \inner{v_1}{w}_F & = & \dim_F \inner{v_1}{v_2}_F & = & 2.
		\end{array}
		\right\}
	\end{equation*}
	Note that we use $\QQ_4$ to 
	denote the restriction of $\QQ_{10}$ on $V_4$. With respect to $\mathcal{B}$, $w$ has the 
	form $[\alpha, \beta, \gamma, \delta]$ for some $\alpha,\beta,\gamma,\delta \in F$.
	By polarising the quadratic form, we find that in this basis the inner product is 
	represented by the matrix
	\begin{equation*}
		B = \begin{bmatrix}[r]
			0 & 0 & 0 & 1 \\
			0 & 0 & -1 & 0 \\
			0 & -1 & 0 & 0 \\
			1 & 0 & 0 & 0
		\end{bmatrix},
	\end{equation*}
	so we find $\inner{v_1}{w} = v_1 B w^{\T} = \delta$, so $\delta = 0$. Next, 
	\mbox{$\QQ_4(w) = -\beta \gamma$}, so either $\beta = 0,\ \gamma \neq 0$ or $\beta \neq 0,\ \gamma = 0$. The 
	case $(\beta, \gamma) = (0,0)$ does not satisfy the condition on dimension, so we do not consider it. 
	
	It follows that either $w = \alpha v_1 + \beta v_2$, $\beta \neq 0$ or $w = \alpha v_1 + \gamma v_3$, $\gamma \neq 0$. 
	Consider an element $h_2 \in \SL_2(F) \circ \SL_2(F)$, which has the following matrix form:
	\begin{equation*}
		[h_2]_{\mathcal{B}} = 
		\begin{bmatrix}
			1 & 0 & 0 & 0 \\
			\lambda & 1 & 0 & 0 \\
			0 & 0 & 1 & 0 \\
			0 & 0 & \lambda & 1
		\end{bmatrix}.
	\end{equation*}
	We have $(v_2^{h_2})^{g h_1^{-1}} = (\lambda v_1 + v_2)^{g h_1^{-1}}$, which is either $(\alpha + \lambda) v_1 + \beta v_2$, or 
	$(\alpha + \lambda) v_1 + \gamma v_3$. Take $\lambda = -\alpha$ to get $v_2^{h_2 g h_1^{-1}}$ to be either 
	$\beta v_2$ for some $\beta \neq 0$, or $\gamma v_3$ for some $\gamma \neq 0$. 
	
	In $\Omega_4^+(F)$ our element $h_2 g h_1^{-1}$ has the following matrix form with respect to the basis $\mathcal{B}$:
	\begin{equation*}
		[h_2 g h_1^{-1}]_{\mathcal{B}} = 
		\left[
	    \begin{array}{c|c|c}
		1 & 0 & 0\  \\ \hline 
		* &\  A\   & 0\  \\ \hline
		* & * & 1\ 
	    \end{array}
	\right],
	\end{equation*}
	where $A$ represents an element of $\Omega_2^+(F)$. Now, $h_2 g h_1^{-1}$ has spinor norm $1$, as well as the element 
	represented by $A$. In the case when $h_2 g h_1^{-1}$ maps $v_2$ to $\beta v_2$, $A$ takes the form
	\begin{equation*}
		A = \begin{bmatrix}
			\beta & 0 \\
			0 & \beta^{-1}
		\end{bmatrix}.
	\end{equation*}
	On the other hand, if $v_2$ is mapped to $\gamma v_3$, then 
	\begin{equation*}
		A = \begin{bmatrix}
			0 & \gamma \\
			\gamma^{-1} & 0
		\end{bmatrix}.
	\end{equation*}
	The latter is of no interest to us, because its determinant is $-1$, and in characteristic $2$ it has the wrong quasideterminant.
	Since also the spinor norm of the element represented by $A$ is $1$, $\beta$ is a square in $F$. Take $\lambda \in F$ such
	that $\lambda^2 = \beta$, and consider the $4\times 4$ matrix $C$:
	\begin{equation*}
		C = \begin{bmatrix}
			1 & 0 & 0 & 0 \\
			0 & \lambda^2 & 0 & 0 \\
			0 & 0 & \lambda^{-2} & 0 \\
			0 & 0 & 0 & 1
		\end{bmatrix}.
	\end{equation*}
	It is easy to see that $C$ represents an element of $\SL_2(F) \circ \SL_2(F)$. Indeed,
	\begin{equation*}
		C = \left( \begin{bmatrix}
			\lambda & 0 \\
			0 & \lambda^{-1}
		\end{bmatrix} \otimes
		\begin{bmatrix}
			1 & 0 \\
			0 & 1
		\end{bmatrix} \right) \cdot
		\left( 
		\begin{bmatrix}
			1 & 0 \\
			0 & 1
		\end{bmatrix}	\otimes
		\begin{bmatrix}
			\lambda^{-1} & 0 \\
			0 & \lambda
		\end{bmatrix}
		 \right)
	\end{equation*}
	Denote by $h_3$ the element of $\SL_2(F) \circ \SL_2(F)$, represented by $C$.
	Now, the matrix representing $h_2 g h_1^{-1} h_3^{-1}$ with respect to the basis $\mathcal{B}$ has ones on the diagonal:
	\begin{equation*}
		[h_2 g h_1^{-1} h_3^{-1}]_{\mathcal{B}} = \begin{bmatrix}
			1 & 0 & 0 & 0 \\
			\varepsilon & 1 & 0 & 0 \\
			\zeta & 0 & 1 & 0 \\
			\eta & \iota & \kappa & 1
		\end{bmatrix},
	\end{equation*}	
	where $\varepsilon,\zeta,\eta,\iota,\kappa \in F$. Denote	 by $w_1$, $w_2$, $w_4$ the images of 
	$v_1$, $v_2$, and $v_4$ respectively, under the action of $h_2 g h_1^{-1} h_3^{-1}$. Since
	$\QQ_4(v_2) = 0$ and $\QQ_4(w_4) = \eta - \iota\kappa$, we get $\eta = \iota \kappa$. 
	Next, $\inner{v_2}{v_4} = 0$ and $\inner{v_2}{w_4} = \varepsilon - \kappa$, so $\varepsilon = \kappa$.
	Finally, $\inner{v_1}{v_4} = 0$ and $\inner{v_1}{w_4} = \zeta - \iota$, so $\zeta = \iota$.
	It turns out that
	\begin{equation*}
		[h_2 g h_1^{-1} h_3^{-1}]_{\mathcal{B}} = \begin{bmatrix}
			1 & 0 & 0 & 0 \\
			\varepsilon & 1 & 0 & 0 \\
			\zeta & 0 & 1 & 0 \\
			\varepsilon \zeta & \zeta & \varepsilon & 1
		\end{bmatrix} = 
		\begin{bmatrix}
			1 & 0 \\
			\zeta & 1 
		\end{bmatrix} \otimes
		\begin{bmatrix}
			1 & 0 \\
			\varepsilon & 1
		\end{bmatrix}.
	\end{equation*}
	Thus, $h_2 g h_1^{-1} h_3^{-1}$ is an element of 
	$\SL_2(F) \circ \SL_2(F)$, and therefore so is $g$.
\end{proof}


% ------------------------------------------------------------------------

%%% Local Variables: 
%%% mode: latex
%%% TeX-master: "../thesis"
%%% End: 
