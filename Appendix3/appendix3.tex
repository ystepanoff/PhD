\chapter{C: $\Omega_4^{-,K}(F)$}
\label{AppC}
\stepcounter{appendixsection}

Let $K$ be a quadratic extension field of $F$ and let $\s$ be the field
automorphism. Define $V_4$ as a vector space over $K$ with the following 
basis $\mathcal{B}$:
\begin{equation}
	\begin{array}{r@{\;}c@{\;}l}
		v_1 & = & [0,1,0,0], \\
		v_2 & = & [0,0,0,1], \\
		v_3 & = & [1,0,0,0], \\
		v_4 & = & [0,0,1,0].
	\end{array}
\end{equation}
Define the quadratic 
form $\QQ_4$ via
\begin{equation}
	\QQ_4([a, b, c, d]) = ad - bc.
\end{equation} 

\begin{lemma}
	\label{lemma:C_omega4minus}
	The marices
	\begin{equation*}
		A_{\lambda}	= \begin{bmatrix}
			1 & 0 & 0 & 0 \\
			-\lambda^{\s} & 1 & 0 & 0 \\
			\lambda & 0 & 1 & 0 \\
			-\lambda\lambda^{\s} & \lambda & -\lambda^{\s} & 1
		\end{bmatrix},\quad 
		B_{\lambda} = \begin{bmatrix}
			1 & \phantom{-}\lambda & -\lambda^{\s} & -\lambda\lambda^{\s} \\
			0 & \phantom{-}1 & 0 & -\lambda^{\s} \\
			0 & \phantom{-}0 & 1 & \lambda \\
			0 & \phantom{-}0 & 0 & 1
		\end{bmatrix}
	\end{equation*}
	generate $\Omega_4^{-,K}(F)$ as $\lambda$ ranges thgrouh $K$.
\end{lemma}

\begin{proof}
We notice that 
	\begin{equation*}
		A_{\lambda} = \begin{bmatrix}
			1 & 0 \\
			\lambda & 1 
		\end{bmatrix} \otimes
		\begin{bmatrix}
			1 & 0 \\
			-\lambda^{\s} & 1
		\end{bmatrix},\quad
		B_{\lambda} = \begin{bmatrix}
			1 & -\lambda^{\s} \\
			0 & 1
		\end{bmatrix} \otimes
		\begin{bmatrix}
			1 & \lambda \\
			0 & 1
		\end{bmatrix},
	\end{equation*}
	where $\otimes$ is the Kronecker product of two matrices. The mapping
	\begin{equation*}
		\begin{bmatrix}
			1 & 0 \\
			\lambda & 1 
		\end{bmatrix} \mapsto
		\begin{bmatrix}
			1 & 0 \\
			\lambda & 1
		\end{bmatrix} \otimes
		\begin{bmatrix}
			1 & 0 \\
			-\lambda^{\s} & 1
		\end{bmatrix},\quad 
		\begin{bmatrix}
			1 & \lambda \\
			0 & 1
		\end{bmatrix} \mapsto
		\begin{bmatrix}
			1 & -\lambda^{\s} \\
			0 & 1
		\end{bmatrix} \otimes
		\begin{bmatrix}
			1 & \lambda \\
			0 & 1
		\end{bmatrix}
	\end{equation*}
	can be extended to a homomorphism $\phi$ which is obviously surjective as $\lambda$
	 ranges
	through the whole field $K$. Its kernel is a subgroup
	\begin{equation*}
		\ker(\phi) = 
		\left\langle
			\begin{bmatrix}
				-1 & 0 \\
				0 & -1
			\end{bmatrix}
		\right\rangle,
	\end{equation*}
	which has order $2$, so we get the action of the group $\PSL_2(K)$ on $V_4$ since
	the matrices
	\begin{equation*}
		\begin{bmatrix}
			1 & 0 \\
			\lambda & 1
		\end{bmatrix},\quad
		\begin{bmatrix}
			1 & \lambda \\
			0 & 1 
		\end{bmatrix}
	\end{equation*}
	generate a group $\SL_2(K)$ by a well-known result.
	Therefore, as $\PSL_2(K) \cong \Omega_4^{-,K}(F)$ (see, for example, \S 7 in \cite{Waerden}), 
	we have the action of $\Omega_4^{-,K}(F)$. To show that the matrices $A_{\lambda}$ and
	$B_{\lambda}$, which generate $\PSL_2(K)$, also generate $\Omega_4^{-,K}(F)$, we first 
	show the transitivity of our copy of $\PSL_2(K)$ on $1$-spaces spanned by isotropic vectors in $V_4$.
	
	Let
	$\langle v \rangle = \langle [\alpha,\beta,\gamma,\delta] \rangle$ be an isotropic point in $V_4$, i.e. with 
	$\alpha \delta - \beta \gamma = 0$. 
	We have
	\begin{equation*}
		\begin{array}{r@{\;}c@{\;}l}
			v A_{\lambda} & = & [\alpha - \lambda^{\s} \beta + \lambda \gamma - \lambda \lambda^{\s} \delta,\  \beta+\lambda\delta,\  
					\gamma-\lambda^{\s}\delta,\  \delta], \\
			
			v B_{\lambda} & = & [\alpha,\  \beta+\lambda\alpha,\  \gamma-\lambda^{\s}\alpha,\  
						\delta - \lambda\lambda^{\s} \alpha - \lambda^{\s}\beta + \lambda\gamma + \delta],
		\end{array}
	\end{equation*}
	so if $\delta \neq 0$ we can make $\beta = \gamma = 0$ by choosing suitable values of $\lambda$ in $A_{\lambda}$, and since the point remains isotropic, we automatically have $\alpha = 0$, so we end up with a point
	of the form $\langle [0,0,0,*] \rangle$. If, on the other hand, $\delta = 0$, then if $\alpha = 0$, we can right-multiply by
	the matrix $A_{\lambda}$ with a suitable value of $\lambda$ to obtain $\alpha \neq 0$. Having $\delta = 0$ and  $\alpha \neq 0$,
	we can make $\beta = \gamma = 0$ as in the case $\delta \neq 0$ by using the matrix $B_{\lambda}$, hence obtaining a point 
	of the form $\langle [*,0,0,0] \rangle$. Finally, to map $\langle [*,0,0,0] \rangle$ to $\langle [0,0,0,*] \rangle$ we can use the element of
	$\PSL_2(K)$ represented by the matrix
	\begin{equation*}
		\begin{bmatrix}[r]
			0 & 0 & 0 & -1 \\
			0 & 0 & 1 & 0 \\
			0 & -1 & 0 & 0 \\
			1 & 0 & 0 & 0
		\end{bmatrix}.
	\end{equation*}
	
	Finally, suppose $g \in \Omega_4^{-,K}(F)$, and since we now have transitivity of $\PSL_2(K)$ on isotropic points,
	we pick an element $h \in \PSL_2(K)$ such that $\langle v_1 \rangle^{g h^{-1}} = \langle v_1 \rangle$, so with respect
	to the chosen basis the element $g h^{-1}$ has the form
	\begin{equation*}
			[g h^{-1}]_{\mathcal{B}} = 
		\left[
	    \begin{array}{c|c|c}
		\lambda & 0 & 0\  \\ \hline 
		* &\  A\   & 0\  \\ \hline
		* & * & \lambda^{-1}\ 
	    \end{array}
	\right],
	\end{equation*}
	where $A$ is chosen so that $g h^{-1}$ has spinor norm $1$. The rest of the proof is rather the same as 
	in Lemma \ref{lemma:B_omega4plus}. 
\end{proof}
% ------------------------------------------------------------------------

%%% Local Variables: 
%%% mode: latex
%%% TeX-master: "../thesis"
%%% End: 
